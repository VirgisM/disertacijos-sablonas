\thispagestyle{empty}                   % no headers and footers
{\fontfamily{ptm}\selectfont
\linespread{1.15}\selectfont
% Nustatome šrifto dydį, plotį, žr. https://tex.stackexchange.com/questions/68745/possible-values-for-fontseries-and-fontshape
\renewcommand\bfdefault{m}% \renewcommand\bfdefault{bc} does not work and changes to m (Medium (normal))

% Visi kintamieji yra nurodyti settings.tex faile / All constants are in settings.tex file
\begin{flushright}
    \thesisDOI \\
    \thesisORCID
\end{flushright}

\begin{center}
	\vspace*{5mm}	
	\begin{flushleft}
         \fontsize{12}{12}\selectfont
	       VILNIUS UNIVERSITY \\
	\end{flushleft}
 
	\vspace{50mm}
	\begin{flushleft}
	   {\fontsize{15}{15}\selectfont  \thesisAuthorName  \\ \MakeUppercase{\thesisAuthorSurname} \par}
    \end{flushleft}

	\vspace{10mm}
	\begin{flushleft}
    	{ \fontsize{21}{21}\selectfont
    	   \thesisTitleEN \par
    	}
    \end{flushleft}

    \vspace{50mm}
    \begin{flushleft}
        \renewcommand\bfdefault{b}
        \fontsize{12}{12}\selectfont
        {\bf DOCTORAL DISSERTATION}\\ 
    \end{flushleft}
    
    \vspace{5mm}
    \begin{flushleft}
        \renewcommand\bfdefault{m}
        \fontsize{12}{12}\selectfont
        \bf
            Natural Sciences, \\ %Mosklo sritis
            Informatics (N 009)  %Mokslo kryptis krypties kodas (kodas skliausteliuose, pvz., (N 001))
    \end{flushleft}
    
    \vspace{6mm}
    \begin{flushleft} 
         \fontsize{9}{9}\selectfont
        \bf VILNIUS \thesisYear
    \end{flushleft} 
\end{center}
}

\newpage
\thispagestyle{empty}                   % no headers and footers

%The dissertation was prepared between 20__ and 20__ (the name of the institution at which the dissertation was completed). The research was supported by (e.g. Research Council of Lithuania, if the doctoral studies were financed from the EU structural funds, or a scholarship was granted for academic accomplishments). 
%\begin{singlespace}
\noindent\nohyphens{The dissertation was prepared between {\thesisPreparationStartYear} and {\thesisYear} at Vilnius University.}
% (In case the doctoral dissertation is defended on an external basis, include the statement ‘The dissertation is defended on an external basis’).
% \vspace{0.5cm}
% \noindent{The dissertation is defended on an external basis.}

\vspace{1cm}
%The research was partially supported by the Research Council of Lithuania (Researcher groups projects Grant), project ... 
%Šita dalis, jeigu buvo papildomas finansavimas

% (If the dissertation is defended on an external basis, write ‘Academic consultant’) Prof. Habil. Dr. Name Surname (Name of institution, Research area, Research field, Field code). (In case the doctoral student had two academic supervisors, indicate the time frame(s) of their supervision).
\noindent {\bf Academic supervisor -- }{ {Prof. Habil. Dr. ...Name Surname..} (Vilnius University, Natural Sciences, Informatics -- N 009)}.\\
\noindent {\bf Academic consultant -- }{ {Prof. Dr. ...Name Surname..} (Vilnius University, Natural Sciences, Informatics -- N 009)}.\\ %(Name of institution, Research area, Research field, Field code).

\vspace{1cm}
\noindent
This doctoral dissertation will be defended in a public meeting of the Dissertation Defence Panel:

\vspace{0.5cm}
\noindent
{\bf Chairman  --} {{Prof. Dr. ...Name Surname...} (Vilnius University, Natural Sciences, Informatics -- N 009)}.\\ %(Name of institution, Research area, Research field, Field code).
{\bf Members:}\\ %[nariai surašomi abėcėlės tvarka pagal pavardes].
{Prof. ...Name Surname...}
(..., Natural Sciences, Informatics -- N 009).\\%(Name of institution, Research area, Research field, Field code)
{Prof. Dr. ...Name Surname...}
(..., Natural Sciences, Informatics -- N 009).\\
{Prof. Habil. Dr. ...Name Surname...} 
(..., Natural Sciences, Informatics -- N 009).\\
{Dr. ...Name Surname...}
(Tallinn University of Technology, Estonia, Natural Sciences, Chemistry -- N 003).\\

\vspace{2cm}
\noindent 
The dissertation shall be defended at a public meeting of the Dissertation Defense Panel at ..... a.m. on ...th .......... 20.. in room 203 of the Institute of Data Science and Digital Technologies of Vilnius University. \\
Address: Akademijos str. 4, LT-04812, Vilnius, Lithuania \\
Tel. +370 5 210 9300; e-mail: info@mii.vu.lt

\vspace{1cm}
\noindent
The text of this dissertation can be accessed at the Library of Vilnius
University, as well on the website of Vilnius University: \\ 
\href{https://www.vu.lt/lt/naujienos/ivykiu-kalendorius}{ \textit{\underline{https://www.vu.lt/lt/naujienos/ivykiu-kalendorius}}}.

%\end{singlespace}

\newpage
\thispagestyle{empty}                   % no headers and footers
{\fontfamily{ptm}\selectfont
\linespread{1.15}\selectfont
\renewcommand\bfdefault{m}
% Visi kintamieji yra nurodyti settings.tex faile / All constants are in settings.tex file
\begin{flushright}
    \thesisDOI \\
    \thesisORCID
\end{flushright}

\begin{center}
	\vspace*{5mm}	
	\begin{flushleft}
         \fontsize{12}{12}\selectfont
	       VILNIUS UNIVERSITAS \\
	\end{flushleft}
 
	\vspace{50mm}
	\begin{flushleft}
	   {\fontsize{15}{15}\selectfont  \thesisAuthorName  \\ \MakeUppercase{\thesisAuthorSurname} \par}
    \end{flushleft}

	\vspace{10mm}
	\begin{flushleft}
    	{ \fontsize{21}{21}\selectfont
    	   \thesisTitleLT \par
    	}
    \end{flushleft}

    \vspace{50mm}
    \begin{flushleft}
        \renewcommand\bfdefault{b}
        \fontsize{12}{12}\selectfont
        {\bf DAKTARO DISERTACIJA}\\ 
    \end{flushleft}
    
    \vspace{5mm}
    \begin{flushleft}
        \renewcommand\bfdefault{m}
        \fontsize{12}{12}\selectfont
        \bf
            Gamtos mokslai, \\ %Mosklo sritis
            Informatika (N 009)  %Mokslo kryptis krypties kodas (kodas skliausteliuose, pvz., (N 001))
    \end{flushleft}
    
    \vspace{6mm}
    \begin{flushleft} 
         \fontsize{9}{9}\selectfont
        \bf VILNIUS \thesisYear
    \end{flushleft} 
\end{center}
}

\newpage
\thispagestyle{empty}                   % no headers and footers

% \begin{singlespace}
\noindent\nohyphens{Disertacija rengta {\thesisPreparationStartYear} -- {\thesisYear} metais Vilniaus universitete.}

% Jei daktaro disertaciją gina eksternas įrašoma 
% \vspace{0.5cm}
% \noindent{Disertacija ginama eksternu.}
%Mokslinius tyrimus rėmė ...(pvz., Lietuvos mokslo taryba, jei doktorantūra buvo finansuojama ES struktūrinių fondų lėšomis ar buvo gauta stipendija už akademinius pasiekimus).

\vspace{1cm}
%Prie mokslinio vadovo ir konsultanto nurodoma: institucijos pavadinimas, mokslų sritis, mokslo kryptis, mokslo krypties kodas; jeigu buvo du doktoranto moksliniai vadovai, nurodomas vadovavimo laikotarpis).
\noindent {\bf Mokslinis (-ė) vadovas (-ė): \\}{ {prof. dr. ...Vardas Pavardė... } (Vilniaus universitetas, gamtos mokslai, informatika -- N 009)}.

\noindent {\bf Mokslinis (-ė) konsultantas (-ė): \\}{ {prof. dr. ...Vardas Pavardė... } (Vilniaus universitetas, gamtos mokslai, informatika -- N 009)}.

\vspace{1cm}
\noindent
Gynimo taryba:  \\
{\bf Pirmininkas (-ė)} {-- {prof. dr. ...Vardas Pavardė...} (Vilniaus universitetas, gamtos mokslai, informatika -- N 009).\\}
{\bf Nariai:}\\ %[nariai surašomi abėcėlės tvarka pagal pavardes].
{prof. ...Vardas Pavardė...}
(..., gamtos mokslai, informatika -- N 009).\\
{prof. dr. ...Vardas Pavardė...}
(..., gamtos mokslai, informatika – N 009).\\
{prof. habil. dr. ...Vardas Pavardė...}
(..., gamtos mokslai, informatika - N 009).\\
{dr. ...Vardas Pavardė...}
(Talino technikos universitetas, Estija, gamtos mokslai, chemija -- N 003).


\vspace{2cm}
\noindent
%Disertacija ginama viešame / uždarame Gynimo tarybos posėdyje 20_ m. _____________ mėn. __ d. ____ val. (institucijos pavadinimas) fakulteto / instituto _______ posėdžių salėje / auditorijoje. Adresas: (gatvė, namo numeris, patalpos numeris, miestas, Lietuva), tel. +370__________ ; el. paštas 
Disertacija ginama viešame Gynimo tarybos posėdyje 20.. m. .......... ... d. ... val. Vilniaus universiteto Duomenų mokslo ir skaitemeninių technologijo intituto  203 auditorijoje. Adresas: Akademijos g. 4, LT-04812, Vilnius, Lietuva, tel. +370 5 210 9300; el. paštas: info@mii.vu.lt.\\

\vspace{1cm}
\noindent
Disertaciją galima peržiūrėti Vilniaus universiteto bibliotekoje ir Vilniaus universiteto interneto svetainėje adresu: 
\href{https://www.vu.lt/lt/naujienos/ivykiu-kalendorius}{ \textit{\underline{https://www.vu.lt/lt/naujienos/} \underline{ivykiu-kalendorius}}}. 

% \end{singlespace}