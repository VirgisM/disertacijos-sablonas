\thispagestyle{empty}                   % no headers and footers
{\fontfamily{Calibri}\selectfont
%\linespread{1.25}\selectfont
\renewcommand\bfdefault{bc}% or \renewcommand\bfdefault{m}
%\renewcommand\seriesdefault{sb}
%\renewcommand\mddefault{sb}
\fontseries{sb}\selectfont

\begin{flushright}
\thesisDOI \\
\thesisORCID \\ 
%https://doi.org/ \\ %Cia suvedama disertcijos doi
%https://orcid.org/ %ORCID id
\end{flushright}
\begin{center}

	\vspace*{5mm}	
	\begin{flushleft}
\renewcommand\bfdefault{bc}
\bf \large
	VILNIUS UNIVERSITY \\
	\end{flushleft}
	
	
	\vspace{50mm}
	\begin{flushleft}
	{\Large \bf  \thesisAuthorName  \\ \MakeUppercase{\thesisAuthorSurname} \par}
    \end{flushleft}

	\vspace{10mm}
	\begin{flushleft}
	{\huge \bf
\fontsize{21}{21}\selectfont
	Dissertation name not in camel case \par
	}
    \end{flushleft}

    \vspace{5mm}
\begin{flushleft}
\renewcommand\bfdefault{b}
  {\bf DOCTORAL DISSERTATION}\\%[-6pt]  
\end{flushleft}
  \vspace{15mm}
%  \vspace{5mm}
  \begin{flushleft}
\renewcommand\bfdefault{bc}
\bf
  Natural Sciences, \\
  Informatics (N 009)
  \end{flushleft}
  %\vspace{60mm}
     \begin{flushleft} 
  		\noindent\rule{3cm}{0.4pt}
     \end{flushleft} 
   \begin{flushleft} \bf
  VILNIUS 2024
   \end{flushleft} 
\end{center}
}
\newpage
\thispagestyle{empty}                   % no headers and footers

\begin{singlespace}
\noindent\nohyphens{This dissertation was written between 2020 and 2024 at Vilnius University.}\\
%The research was partially supported by the Research Council of Lithuania (Researcher groups projects Grant), project ... 
%Sita dalis, jeigu buvo papildomas finansavimas
\vspace{1cm}

\noindent {\bf Academic supervisor: \\}{ {\bf Prof. Dr. ...} (Vilnius University, Natural Sciences, Informatics -- N 009)}.

%\iffalse
\vspace{2cm}
\noindent
Dissertation Defence Panel: \\
{Chair  --} {{\bf Prof. Dr. ...} (Vilnius University, Natural Sciences, Informatics -- N 009)}.\\
Members:\\ %[nariai surašomi abėcėlės tvarka pagal pavardes].
{\bf Prof. ... }
(..., Natural Sciences, Informatics -- N 009).\\
{\bf Prof. Dr. ...}
(..., Natural Sciences, Informatics -- N 009).\\
{\bf Prof. Habil. Dr. ...} 
(..., Natural Sciences, Informatics -- N 009).\\
{\bf Dr. ...}
(Tallinn University of Technology, Estonia, Natural Sciences, Chemistry -- N 003).\\


\vspace{2cm}

\noindent
The dissertation shall be defended at a public meeting of the Dissertation
Defense Panel at 10 a.m. on 6th October 2024 at the Institute of Computer Science of Vilnius University. Address: Didlaukio str. 47, LT-08303, Vilnius, Lithuania, 
tel. +370 5 219 5000; e-mail: mif@mif.vu.lt \\

\vspace{1cm}
\noindent
The text of this dissertation can be accessed at the Library of Vilnius
University and on the website of Vilnius
University:\\ \href{ www.vu.lt/lt/naujienos/ivykiu-kalendorius}{ www.vu.lt/lt/naujienos/ivykiu-kalendorius}.
%\fi



\end{singlespace}

\newpage
\thispagestyle{empty}                   % no headers and footers
{\fontfamily{Calibri}\selectfont
\begin{flushright}
https://doi.org/ \\
https://orcid.org/ %ORCID id
\end{flushright}
\begin{center}
	\vspace*{5mm}	
	   \begin{flushleft} 
	\renewcommand\bfdefault{bc} 
	\bf \large
	VILNIAUS UNIVERSITETAS \\
	   \end{flushleft} 	

	
	\vspace{50mm}
	
	\begin{flushleft}
\renewcommand\bfdefault{bc} \bf
	{\Large Rokas\\ ASTRAUSKAS\par}
\end{flushleft}
	\vspace{10mm}
	\begin{flushleft}
	{\huge 
    \renewcommand\bfdefault{bc}
    \bf
    Reakcijos-difuzijos proces\k{u} kompiuterinis modeliavimas elektrochemin\.{e}je mikroskopijoje ir l\k{a}steli\k{u} sferoiduose	
	%REAKCIJOS-DIFUZIJOS PROCES\k{U} KOMPIUTERINIS MODELIAVIMAS ELEKTROCHEMIN\.{E}JE MIKROSKOPIJOJE IR L\k{A}STELI\k{U} SFEROIDUOSE
	}
	\end{flushleft}
  \vspace{5mm}
  	\begin{flushleft}
  	{\bf DAKTARO DISERTACIJA}\\%[-6pt]  
  	\end{flushleft}
  \vspace{15mm}
    	\begin{flushleft}
\renewcommand\bfdefault{bc} \bf
  Gamtos mokslai,\\
  informatika (N 009)
  	\end{flushleft}
  %\vspace{60mm}
     \begin{flushleft} 
	\noindent\rule{3cm}{0.4pt}
\end{flushleft} 
\begin{flushleft} 
\renewcommand\bfdefault{bc} \bf
	VILNIUS 2021
\end{flushleft} 
\end{center}
}
\newpage
\thispagestyle{empty}                   % no headers and footers

\begin{singlespace}
\noindent\nohyphens{Disertacija rengta 2014 -- 2018 metais Vilniaus universitete.}
%Mokslinius tyrimus rėmė ...
\vspace{1cm}

\noindent {\bf Mokslinis vadovas: \\}{ {\bf prof. dr. ...} (Vilniaus universitetas, gamtos mokslai, informatika -- N 009)}.

%\iffalse
\vspace{2cm}
\noindent
Gynimo taryba:  \\
{Pirmininkė --} {{\bf prof. dr. ...} (Vilniaus universitetas, gamtos mokslai, informatika -- N 009).\\}
Nariai:\\ %[nariai surašomi abėcėlės tvarka pagal pavardes].
{\bf prof. habil. dr. ...}
(..., gamtos mokslai, informatika -- N 009).\\
{\bf prof. dr. ...}
(..., gamtos mokslai, informatika – N 009.)\\
{\bf prof. ...}
(..., gamtos mokslai, informatika - N 009).\\
{\bf dr. ...}
(Talino technikos universitetas, Estija, gamtos mokslai, chemija -- N 003).



\vspace{2cm}
\noindent
Disertacija ginama viešame Gynimo tarybos posėdyje 2021 m. spalio 6 d. 10 val. Vilniaus universiteto Matematikos ir informatikos fakulteto Informatikos instituto 211 auditorijoje. Adresas: Didlaulio g. 47, LT-08303, Vilnius, Lietuva, tel. +370 5 219 5000; el. paštas: mif@mif.vu.lt.\\

\vspace{1cm}
\noindent
Disertaciją galima peržiūrėti Vilniaus universiteto bibliotekoje ir Vilniaus universiteto interneto svetainėje adresu: \href{ www.vu.lt/lt/naujienos/ivykiu-kalendorius}{ www.vu.lt/lt/naujienos/ ivykiu-kalendorius}. 
%\fi



\end{singlespace}

