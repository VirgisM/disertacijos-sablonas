%%%%%%%%%%%%%%%%%%%%%%%%%%%%%%%%%%%%%%%%%%%%%%%%%%%%%%%%%%%%%%%%%%%%%%%%%%%%%%%%%%%%%%%%%%%%%%
%% [file: index.tex, started: 25-Aug-2005]
%%
%% PhD Thesis - top level LaTeX source file.
%%
%% DESCRIPTION
%%   This file includes Index chapter of the PhD Thesis. 
%%   USe command \index{keyword} to add "keyword" to the index
%%
%% CHANGES
%%   2005.08.25  *  Started.
%%   2008.03.18  *  Adapted to IZ.
%%   2024.08.30  *  Fixed to fit VU dissertation template
%% I overleaf you have to rebuild document from scratch to show renewed index

% \chapter{Index}
% \label{chapter:Index}
% \addtocontents{toc}{\protect\enlargethispage{2\baselineskip}}
\phantomsection
% \addcontentsline{toc}{chapter}{\numberline{}Index} % \numberline{} atitraukia į šalį
\addcontentsline{toc}{chapter}{\MakeUppercase{Index / Rodyklė}}

%% Show index items in two-column style sorted by name
%To add something to the index please use command \index{keyword}

% \makeatletter
% \def\imki@columns{2}
% \makeatother
\def\indexname{Index / Rodyklė}  %Rename the title of the index

% To change various paramettres
% \renewcommand{\see}[2]{\emph{\seename}~{#1 #2}}
% \renewcommand{\see}[2]{\emph{\alsoname}~{#1 #2}}
% Lituanization "see", and "see also"
% \renewcommand{\seename}{žr.}
% \renewcommand{\alsoname}{žr. taip pat}

{\raggedright
    \printindex
}
