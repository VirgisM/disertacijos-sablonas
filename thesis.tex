\documentclass[11pt,b5paper,twoside,openany,final]{book}

\usepackage{ifpdf}

%%%%%%%%%%%%%%%%%%%%%%%%%%%%%%%%%%%%%%%%%%%%%%%%%%%%%%%%%%%%%%%%%%%%%%%%%%%%%%%%%%%%%%%%
%% Used to define various parameters used globally in the dissertation

%% The author name and surname
\def \thesisAuthorName {Vardas}
\def \thesisAuthorSurname {Pavardė}

%% DOI number of the dissertation / DOI numeris (suteikiamas atsiuntus disertaciją spausdinti)
\def \thesisDOI {https://doi.org/}

%% If you do not have ORCID please create it using https://orcid.org / Jeigu ORCID dar neturite, nemokamai galite susikurti https://orcid.org/)
\def \thesisORCID {https://orcid.org/0000-0001-2345-6789} 

%% Year of start Ph.D and  year of writing this dissertation
\def \thesisPreparationStartYear {2020}
\def \thesisYear {2024}

%% The title of dissetaion in English (thesisTitleEN) and in Lithuanian (thesisTitleLT)
%% Įveskite daktaro disertacijos pavadinmą angliškai (thesisTitleEN) ir lietuviškai (thesisTitleLT)
\def \thesisTitleEN {Title of the doctoral dissertation}
\def \thesisTitleLT {Daktaro disertacijos pavadinimas} 

%% Renaming of Table of Contents, List of Tables, and List of Figures
%% Turinio lentelių pervadinimas
\def \tableOfContentsName {Table of Contents / Turinys}
\def \listOfTableName {List of Tables / Lentelių sąrašas}
\def \listOfFiguresName {List of Figures / Paveikslų sąrašas}

%% Define editors for Enlish version and Lithuanian version
%% Disertacijos ir santraukos redaktoriai
\def \thesisEditorEN {Name Surname}
\def \thesisEditorLT {Name Surname}
 %Various settings
% [file: layout.tex, started: 18-March-2008]
%
% NOTES
%   This macro file contains page layout info for Ph.D. thesis
%
% CHANGES
%   2008.03.18	*	Started.

% ----- layout parameters ------------------------------------------------
\ifpdf
	\usepackage[dvips=false, pdftex=true, vtex=false]{geometry}
\else
	\usepackage[dvips=true, pdftex=false, vtex=false]{geometry}
\fi

% VU requirements 2021 m.
%puslapio dydis – B5 (17x24 cm),
%apatinė paraštė 2 cm, viršutinė paraštė 1,5 cm, kairė ir dešinė paraštės turi būti 2,5 cm,
%tarpai tarp eilučių – 1,15 intervalo,  (Rokas.: siuo metu paliktas defaultinis)
%pirmojo lygmens antraštė – 12 pt. Times New Roman; likusios paantraštės - 11 pt. Times New Roman,
%teksto šriftas – 11 pt. Times New Roman,
\geometry{%
 	b5paper,
    paperwidth=170mm, 
    paperheight=240mm,
 	bindingoffset=0pt,
 	centering,
 	hmargin=25mm,
    bottom=20mm,
    top=15mm%,
 	%includehead,
 	%includefoot
}  
  
\raggedbottom                           % height of text may vary per page

%\iffalse
\newlength{\oldparindent}
\newlength{\oldparskip}

\newcommand{\IZParagraph}{%
    \setlength{\oldparindent}{\parindent}
    \setlength{\oldparskip}{\parskip}
    \setlength{\parindent}{5mm}            % no indent
    \setlength{\parskip}{2ex plus 0.5ex minus 0.2ex}    % space between paragraphs
}

\newcommand{\restoreParagraph}{%
    \setlength{\parindent}{\oldparindent}
    \setlength{\parskip}{\oldparskip}
}
%\fi

%% VU requirements
%% 1.15 linespacing, bet paliktas defaultinis
%\linespread{1.15}
\usepackage{setspace}
%\onehalfspacing


\ifpdf
    \usepackage[pdftex]{hyperref,graphicx} % pdf references support

    \hypersetup{%
    	pdfauthor   = {\thesisAuthorName \ \thesisAuthorSurname},
    	pdftitle    = {\thesisTitleEN},
    	pdfsubject  = {PhD Thesis},
    	pdfkeywords = {\thesisTitleLT},
    	pdfcreator  = {LaTeX with hyperref package},
    	pdfproducer = {pdflatex},
    	breaklinks  = {true},
    	pdfstartview = {FitH},
    	bookmarksnumbered = {true},
    	bookmarksopen = {true},	
    	bookmarksopenlevel = 2,
    	plainpages={false},
    	pdfpagetransition={Dissolve},	
    	colorlinks = {true}, %defaultinis, padaro spalvotus stačiakampius ant visų linkų. Uždėjus true, bus tik spalvos ant skaicių. Turbut gražiau atrodo tik spalvos.
    	%Norint nespalvotu nuorodu į formules, literatura, etc:
    	linkcolor = {black}, %Vidinių nuorodų spalava
    	citecolor = {black}, %Citavimų spalva
    	filecolor = {black}, 
        urlcolor  = {black}, % Spalva išorinių nuorodų
    	%pilkos atspalviai:
    	% linkcolor = [rgb]{0.3, 0.3, 0.3},
    	%citecolor = [rgb]{0.6, 0.6, 0.6},
    	%filecolor = [rgb]{0.3, 0.3, 0.3},
    	%runcolor = [rgb]{0.3, 0.3, 0.3},
    }
\else
    \usepackage[dvips,ps2pdf]{hyperref,graphicx} % pdf references support
\fi

% ----- packages ---------------------------------------------------------
\usepackage[immediate]{silence} %package to silence some warnings
\WarningFilter[temp]{latex}{Command \underbar has changed.} % silence sectsty package the warning
\WarningFilter[temp]{latex}{Command \underline has changed.} % silence sectsty package the warning

%------ įkėliau iš preambule į santrauka
%\usepackage[L7x]{fontenc}
%\usepackage[utf8]{inputenc} % Accept different input encodings
\usepackage[T1]{fontenc} %Font encoding. The T1 font encoding is an 8-bit encoding and uses fonts that have 256 glyphs. 

\usepackage{times} %It does set \rmdefault to Times, \sfdefault to Helvetica, and \ttdefault to Courier

\usepackage{aeguill} %The package enables the user to add guillemets from several source (Polish cmr, Cyrillic cmr, lasy and ec) to the ae fonts.

\usepackage[lithuanian, british]{babel} %This package manages culturally-determined typographical (and other) rules for a wide range of languages.
\newcommand{\british}{\selectlanguage{british}}
\newcommand{\lithuanian}{\selectlanguage{lithuanian}}

\usepackage{amssymb}					% Extra AMS math symbols
\usepackage{amsfonts}					% Extra AMS math symbols
\usepackage{amsmath}					% AMS 
\usepackage[mathscr]{euscript}          % This file sets up some font shape definitions to use the Euler script symbols in math mode. 

\usepackage{xcolor}				        % Colors support
\renewcommand{\topfraction}{0.85}	    %% 85% of page can contain figure
\renewcommand{\textfraction}{0.1}	    %% 10% of page can contain text
\renewcommand{\floatpagefraction}{0.75}	%% 75% of page should be figure to be only float page

\usepackage[square,comma,numbers,sort&compress]{natbib}	% Citations
\setlength{\bibsep}{0ex}

\usepackage{fancyhdr}		    % Fancy headings and footers
\usepackage{sectsty}			% Change sections fonts, raises two warnings			

\setcounter{secnumdepth}{3}	    % Depth of enumerated sections

\ifpdf
	\usepackage{microtype}		% Experimental https://texdoc.org/serve/microtype/0
\else
\fi

\usepackage{makeidx}			% index package
\makeindex						% generate index entries
%\usepackage{fix2col}			% fix two-column marks
%%%%%%%%%%%%%%%%%%%%%%%%%%%%%%%%%%%%%%%%%%%%%%%%%%%%%%%%%%%%%%%%%%%%%%%%%%%%%%%%%%%%%%%%%%%%%%
%% [file: index.tex, started: 25-Aug-2005]
%%
%% PhD Thesis - top level LaTeX source file.
%%
%% DESCRIPTION
%%   This file includes Index chapter of the PhD Thesis. 
%%   USe command \index{keyword} to add "keyword" to the index
%%
%% CHANGES
%%   2005.08.25  *  Started.
%%   2008.03.18  *  Adapted to IZ.
%%   2024.08.30  *  Fixed to fit VU dissertation template

\chapter{Index}
\label{chapter:Index}
% \addtocontents{toc}{\protect\enlargethispage{2\baselineskip}}
\phantomsection
% \addcontentsline{toc}{chapter}{\numberline{}Index} % \numberline{} atitraukia į šalį
\addcontentsline{toc}{chapter}{\MakeUppercase{Index}}

%% Show index items in two-column style sorted by name
%To add something to the index please use command \index{keyword}
{\raggedright
    \printindex
}
	    % macros for dictionary like index			

\usepackage{palatino}			% IZ selection, font package

\usepackage{verbatim}           %IZ: multiline komentams 

\usepackage{paralist}           % Provides enumerate and itemize environments that can be used within paragraphs. https://ctan.org/pkg/paralist
\usepackage{tabularx}           % More advanced tables
\usepackage{longtable}          % For long tables used in acronyms, notations, vocabulary
\usepackage{booktabs}           % The package enhances the quality of tables. https://ctan.org/pkg/booktabs
\usepackage{multirow}           % Create tabular cells spanning multiple rows. https://ctan.org/pkg/multirow
\setlength{\belowcaptionskip}{5pt}

\usepackage{rotating}	        % Rotation tools, including rotated full-page floats https://ctan.org/pkg/rotating

\usepackage{epigraph}	        % A package for typesetting epigraphs. Epigraphs are the pithy quotations often found at the start (or end) of a chapter. https://ctan.org/pkg/epigraph

\usepackage{hyphenat}           %Disable/enable hypenation (žodžių perkėlimui įnaują       eilutę). By default is enabled in babel package. Used in title
 
\usepackage{pdfpages}           % Prikabina išorinius pdf failus prie latex pdf failo

\usepackage{array}              % Extending the array and tabular environments. https://ctan.org/pkg/array

\usepackage{float}              % Improved interface for floating objects. https://ctan.org/pkg/float

\usepackage{siunitx}   %paketas skaiciams ir SI vienetams vaizduoti. Galima ir nenaudoti, bet aukstesnis lygis, naudojamo pavyzdys \qty{67890}{\degree}
%\usepackage{textcomp}  %senas paketas, bet reikalingas kad nemestu kai kuriu siunitx klaidu
%\sisetup{load-configurations = abbreviations}

\setlength{\headheight}{19pt}  %Kažkokia problema su headheight

%Reikia sablonui. Galima ištrinti
\usepackage[version=4]{mhchem} %Typeset chemical formulae/equations https://ctan.org/pkg/mhchem
\DeclareSIUnit \uM{\micro M}
\DeclareSIUnit \mM{\milli M}

\usepackage{tocloft}           %Control table of contents, figures, etc https://www.ctan.org/pkg/tocloft

%------ Commands -----------
\renewcommand{\contentsname}{Table of Contents}
\renewcommand{\cftchapleader}{\cftdotfill{\cftdotsep}} % for chapters adds dots in table of contents
\renewcommand{\cftchapfont}{\normalfont} %Change chapter font to normal in table of contents
\renewcommand{\cftchappagefont}{\normalfont} %Change chapter page number font to normal in table of contents

% ----- macros ---------------------------------------------------------
\input{macros/iz_macros.tex}
% \newcommand{\IZheading}[1]{\scriptsize{\textit{#1}}}

\pagestyle{fancy}
\renewcommand{\chaptermark}[1]{\markboth{\thechapter.\ #1}{}}
\renewcommand{\sectionmark}[1]{\markright{\thesection.\ #1}}
% \renewcommand\thechapter{\arabic{section}}
% \renewcommand\thesection{\arabic{section}}

% \def\thechapter{\arabic{chapter}}

% \newcommand{\IZThesisHeadings}{%
% 	\fancyhead{}\fancyhead[LE]{\IZheading{\IZheading{\leftmark}}}\fancyhead[RO]{\IZheading{\rightmark}}
% 	\fancyfoot{}
% 	\fancyfoot[c]{\thepage}
% 	\fancypagestyle{plain}{%
% 		\fancyhf{}
% 		\fancyfoot[c]{\thepage}
% 		\renewcommand{\headrulewidth}{0pt}
% 	}
% }

\makeatletter
\def\cleardoublepage{%
	\clearpage\if@twoside \ifodd\c@page\else%
	\hbox{}%
	\thispagestyle{empty}%
	\newpage%
	\if@twocolumn\hbox{}\newpage\fi\fi\fi
}

\makeatother
    
\makeatletter	
\newcommand{\itemhdr}[1]{\par\hangindent 30\p@\relax{#1}\markboth{#1}{#1}}
\makeatother

\newcommand{\indexmarks}{%
	\fancyhead[R]{\IZheading{\leftmark}}
	\fancyhead[L]{\IZheading{\rightmark}}
}

% documentation is available here https://www.overleaf.com/learn/latex/Headers_and_footers#Notes_on_the_commands_used
\newcommand{\parammarks}[1]{
	\fancyhead{}                      %clears the settings for the headers
	\fancyhead[RO,LE]{l}               %uses the header locations RO (RightOdd) and LE (LeftEven) to                                    place the content
	\fancyhead[RO,LE]{\IZheading{#1}} %uses the header locations RO (RightOdd) and LE (LeftEven) to                                    place the content \IZheading{#1}
	\fancyfoot{}                      %clears the settings for the footers
	\fancyfoot[c]{\thepage}           %uses the footer locations center (c) to place the content \                                        of \thepage
	\fancypagestyle{plain}{%  just add page numer at the center of page footer 
		\fancyhf{}	
		\fancyfoot[c]{\thepage}
	\renewcommand{\headrulewidth}{0pt}%
	}
} % Šablone labai parastas heading


%-------Usefull commands --------------
% \k{I} - uždeda nosines raides

\addto\captionsbritish{\renewcommand{\contentsname}{Table of Contents}}

\begin{document}

\frontmatter

\thispagestyle{empty}                   % no headers and footers
{\fontfamily{Calibri}\selectfont
%\linespread{1.25}\selectfont
\renewcommand\bfdefault{bc}% or \renewcommand\bfdefault{m}
%\renewcommand\seriesdefault{sb}
%\renewcommand\mddefault{sb}
\fontseries{sb}\selectfont

\begin{flushright}
\thesisDOI \\
\thesisORCID \\ 
%https://doi.org/ \\ %Cia suvedama disertcijos doi
%https://orcid.org/ %ORCID id
\end{flushright}
\begin{center}

	\vspace*{5mm}	
	\begin{flushleft}
\renewcommand\bfdefault{bc}
\bf \large
	VILNIUS UNIVERSITY \\
	\end{flushleft}
	
	
	\vspace{50mm}
	\begin{flushleft}
	{\Large \bf  \thesisAuthorName  \\ \MakeUppercase{\thesisAuthorSurname} \par}
    \end{flushleft}

	\vspace{10mm}
	\begin{flushleft}
	{\huge \bf
\fontsize{21}{21}\selectfont
	Dissertation name not in camel case \par
	}
    \end{flushleft}

    \vspace{5mm}
\begin{flushleft}
\renewcommand\bfdefault{b}
  {\bf DOCTORAL DISSERTATION}\\%[-6pt]  
\end{flushleft}
  \vspace{15mm}
%  \vspace{5mm}
  \begin{flushleft}
\renewcommand\bfdefault{bc}
\bf
  Natural Sciences, \\
  Informatics (N 009)
  \end{flushleft}
  %\vspace{60mm}
     \begin{flushleft} 
  		\noindent\rule{3cm}{0.4pt}
     \end{flushleft} 
   \begin{flushleft} \bf
  VILNIUS 2024
   \end{flushleft} 
\end{center}
}
\newpage
\thispagestyle{empty}                   % no headers and footers

\begin{singlespace}
\noindent\nohyphens{This dissertation was written between 2020 and 2024 at Vilnius University.}\\
%The research was partially supported by the Research Council of Lithuania (Researcher groups projects Grant), project ... 
%Sita dalis, jeigu buvo papildomas finansavimas
\vspace{1cm}

\noindent {\bf Academic supervisor: \\}{ {\bf Prof. Dr. ...} (Vilnius University, Natural Sciences, Informatics -- N 009)}.

%\iffalse
\vspace{2cm}
\noindent
Dissertation Defence Panel: \\
{Chair  --} {{\bf Prof. Dr. ...} (Vilnius University, Natural Sciences, Informatics -- N 009)}.\\
Members:\\ %[nariai surašomi abėcėlės tvarka pagal pavardes].
{\bf Prof. ... }
(..., Natural Sciences, Informatics -- N 009).\\
{\bf Prof. Dr. ...}
(..., Natural Sciences, Informatics -- N 009).\\
{\bf Prof. Habil. Dr. ...} 
(..., Natural Sciences, Informatics -- N 009).\\
{\bf Dr. ...}
(Tallinn University of Technology, Estonia, Natural Sciences, Chemistry -- N 003).\\


\vspace{2cm}

\noindent
The dissertation shall be defended at a public meeting of the Dissertation
Defense Panel at 10 a.m. on 6th October 2024 at the Institute of Computer Science of Vilnius University. Address: Didlaukio str. 47, LT-08303, Vilnius, Lithuania, 
tel. +370 5 219 5000; e-mail: mif@mif.vu.lt \\

\vspace{1cm}
\noindent
The text of this dissertation can be accessed at the Library of Vilnius
University and on the website of Vilnius
University:\\ \href{ www.vu.lt/lt/naujienos/ivykiu-kalendorius}{ www.vu.lt/lt/naujienos/ivykiu-kalendorius}.
%\fi



\end{singlespace}

\newpage
\thispagestyle{empty}                   % no headers and footers
{\fontfamily{Calibri}\selectfont
\begin{flushright}
https://doi.org/ \\
https://orcid.org/ %ORCID id
\end{flushright}
\begin{center}
	\vspace*{5mm}	
	   \begin{flushleft} 
	\renewcommand\bfdefault{bc} 
	\bf \large
	VILNIAUS UNIVERSITETAS \\
	   \end{flushleft} 	

	
	\vspace{50mm}
	
	\begin{flushleft}
\renewcommand\bfdefault{bc} \bf
	{\Large Rokas\\ ASTRAUSKAS\par}
\end{flushleft}
	\vspace{10mm}
	\begin{flushleft}
	{\huge 
    \renewcommand\bfdefault{bc}
    \bf
    Reakcijos-difuzijos proces\k{u} kompiuterinis modeliavimas elektrochemin\.{e}je mikroskopijoje ir l\k{a}steli\k{u} sferoiduose	
	%REAKCIJOS-DIFUZIJOS PROCES\k{U} KOMPIUTERINIS MODELIAVIMAS ELEKTROCHEMIN\.{E}JE MIKROSKOPIJOJE IR L\k{A}STELI\k{U} SFEROIDUOSE
	}
	\end{flushleft}
  \vspace{5mm}
  	\begin{flushleft}
  	{\bf DAKTARO DISERTACIJA}\\%[-6pt]  
  	\end{flushleft}
  \vspace{15mm}
    	\begin{flushleft}
\renewcommand\bfdefault{bc} \bf
  Gamtos mokslai,\\
  informatika (N 009)
  	\end{flushleft}
  %\vspace{60mm}
     \begin{flushleft} 
	\noindent\rule{3cm}{0.4pt}
\end{flushleft} 
\begin{flushleft} 
\renewcommand\bfdefault{bc} \bf
	VILNIUS 2021
\end{flushleft} 
\end{center}
}
\newpage
\thispagestyle{empty}                   % no headers and footers

\begin{singlespace}
\noindent\nohyphens{Disertacija rengta 2014 -- 2018 metais Vilniaus universitete.}
%Mokslinius tyrimus rėmė ...
\vspace{1cm}

\noindent {\bf Mokslinis vadovas: \\}{ {\bf prof. dr. ...} (Vilniaus universitetas, gamtos mokslai, informatika -- N 009)}.

%\iffalse
\vspace{2cm}
\noindent
Gynimo taryba:  \\
{Pirmininkė --} {{\bf prof. dr. ...} (Vilniaus universitetas, gamtos mokslai, informatika -- N 009).\\}
Nariai:\\ %[nariai surašomi abėcėlės tvarka pagal pavardes].
{\bf prof. habil. dr. ...}
(..., gamtos mokslai, informatika -- N 009).\\
{\bf prof. dr. ...}
(..., gamtos mokslai, informatika – N 009.)\\
{\bf prof. ...}
(..., gamtos mokslai, informatika - N 009).\\
{\bf dr. ...}
(Talino technikos universitetas, Estija, gamtos mokslai, chemija -- N 003).



\vspace{2cm}
\noindent
Disertacija ginama viešame Gynimo tarybos posėdyje 2021 m. spalio 6 d. 10 val. Vilniaus universiteto Matematikos ir informatikos fakulteto Informatikos instituto 211 auditorijoje. Adresas: Didlaulio g. 47, LT-08303, Vilnius, Lietuva, tel. +370 5 219 5000; el. paštas: mif@mif.vu.lt.\\

\vspace{1cm}
\noindent
Disertaciją galima peržiūrėti Vilniaus universiteto bibliotekoje ir Vilniaus universiteto interneto svetainėje adresu: \href{ www.vu.lt/lt/naujienos/ivykiu-kalendorius}{ www.vu.lt/lt/naujienos/ ivykiu-kalendorius}. 
%\fi



\end{singlespace}

	
		

\cleardoublepage
\restoreParagraph
\pagestyle{plain}
%\IZThesisHeadings
%%%%%%%%%%%%%%%%%%%%%%%%%%%%%%%%%%%%%%%%%%%%%%%%%%%%%%%%%%%%%%%%%%%%%%%%%%%%%%%%%%%%%%%%%%%%%%
%% This file includes Table of Contents

\phantomsection %The \phantomsection command is needed to create a link to a place in the document that is not a figure, equation, table, section, subsection, chapter, etc.

\tableofcontents % Table of Contents

%\IZThesisHeadings

%-------------- Prasideda pagrindine disertacijos dalis  ------------------
%\chapter*{Notation - Žymėjimai}
\label{cha:notation}
\addcontentsline{toc}{chapter}{\MakeUppercase{Notation - Žymėjimai}}
%\addcontentsline{toc}{chapter}{\numberline{}Notation}
%\parammarks{Notation}

\noindent
 \begin{longtable}[l]{ p{3cm} p{8cm} } %Todal available width is 11cm
    $A$         & amplitude caused by the feeding screw \\
	$a$         & acceleration of a mass change \\
	$\alpha_1$  & the weight of distance in space \\
	$\alpha_2$  & the weight of distance in time \\
    \label{tab:notation}
\end{longtable}


%OLD style TODO remove
% n{compactitem}[]
% 	\item $A$ amplitude caused by the feeding screw
% 	\item $a$ acceleration of a mass change
% 	\item $\alpha_1$ the weight of distance in space
% 	\item $\alpha_2$ the weight of distance in time
% \end{compactitem}   %Pazymejimai

\mainmatter

%\IZThesisHeadings
\pagestyle{plain}
\IZParagraph


\chapter*{Introduction}
\label{cha:intro}
\addcontentsline{toc}{chapter}{\MakeUppercase{Introduction}} 

\section*{Research area}

Lorem ipsum dolor sit amet, consectetur adipiscing elit. Duis semper hendrerit faucibus. Donec mauris quam, condimentum quis velit et, sodales luctus arcu. Etiam eget rhoncus nunc, in tempus urna. Etiam dignissim quam libero, et aliquam urna tincidunt ac. Aliquam erat volutpat. Aliquam non urna nulla. Aliquam sodales porta tristique. Suspendisse efficitur ante non elit consectetur, et tincidunt nisl ultrices. Proin mollis eleifend lacus, ut fermentum justo porta a.
Fusce convallis, ipsum sed suscipit tincidunt, lectus erat rhoncus lacus, quis faucibus dolor felis auctor felis. Suspendisse iaculis mi nec bibendum lobortis. Sed placerat eget tellus ut rutrum. Mauris finibus leo arcu, ut consectetur nisl condimentum eget. Donec bibendum eros orci, vel aliquam magna efficitur ac. Nulla sit amet luctus lorem, at viverra risus. Suspendisse scelerisque in arcu at ultrices. Aliquam nunc magna, aliquet quis accumsan quis, interdum id libero.
Praesent nisi neque, aliquam eu tempus vitae, faucibus nec est. Nam eu est vel risus aliquam luctus. Maecenas a urna at risus pharetra rutrum. Class aptent taciti sociosqu ad litora torquent per conubia nostra, per inceptos himenaeos. Aenean mauris nulla, commodo vitae dignissim nec, sagittis at tellus. Suspendisse potenti. Nullam a felis hendrerit, iaculis nisi volutpat, vehicula lacus. Ut ac sapien risus. Fusce commodo odio et fringilla egestas. Mauris at sollicitudin neque. In non maximus dui, ut rhoncus dolor. Etiam bibendum porta sem.


\addcontentsline{toc}{section}{Research area} 


\section*{Actuality}

Lorem ipsum dolor sit amet, consectetur adipiscing elit. Duis semper hendrerit faucibus. Donec mauris quam, condimentum quis velit et, sodales luctus arcu. Etiam eget rhoncus nunc, in tempus urna. Etiam dignissim quam libero, et aliquam urna tincidunt ac. Aliquam erat volutpat. Aliquam non urna nulla. Aliquam sodales porta tristique. Suspendisse efficitur ante non elit consectetur, et tincidunt nisl ultrices. Proin mollis eleifend lacus, ut fermentum justo porta a.


\addcontentsline{toc}{section}{Actuality} 


\section*{Research objective}

Lorem ipsum dolor sit amet, consectetur adipiscing elit. Duis semper hendrerit faucibus. Donec mauris quam, condimentum quis velit et, sodales luctus arcu. Etiam eget rhoncus nunc, in tempus urna. Etiam dignissim quam libero, et aliquam urna tincidunt ac. Aliquam erat volutpat. Aliquam non urna nulla. Aliquam sodales porta tristique. Suspendisse efficitur ante non elit consectetur, et tincidunt nisl ultrices. Proin mollis eleifend lacus, ut fermentum justo porta a.


\addcontentsline{toc}{section}{Research objective} 

% \titlespacing*{\chapeter} {0pt}{3.5ex plus 1ex minus .2ex}{2.3ex plus .2ex}
% \titlespacing*{\section} {0pt}{3.5ex plus 1ex minus .2ex}{2.3ex plus .2ex}
% \titlespacing*{\subsection} {0pt}{3.25ex plus 1ex minus .2ex}{1.5ex plus .2ex}
% \titlespacing*{\subsubsection}{0pt}{3.25ex plus 1ex minus .2ex}{1.5ex plus .2ex}
% \titlespacing*{\paragraph} {0pt}{3.25ex plus 1ex minus .2ex}{1em}

\def \mychapter {
\titleformat{\chapter}[display]
%{chapter style}{chapter text} {space before chapter title} {chapter title format}
{\normalfont}{-}{-10mm}{\thechapter . \centering\normalfont\MakeUppercase}
} 



\setcounter{chapter}{0}
\chapter{\MakeUppercase{Literature review / Literatūros apžvalga}} %Make it upper case to appear in TOC correctly
\label{cha:review}
% You should rename the title of this chapter to fit your dissertation domain
% Skyriaus pavadinimas parenkamas taip, kad atitiktų apžvalgą disertacijos srityje

Lorem ipsum dolor sit amet, consectetur adipiscing elit. Duis semper hendrerit faucibus. Donec mauris quam, condimentum quis velit et, sodales luctus arcu. Etiam eget rhoncus nunc, in tempus urna. Etiam dignissim quam libero, et aliquam urna tincidunt ac. Aliquam erat volutpat. Aliquam non urna nulla. Aliquam sodales porta tristique. Suspendisse efficitur ante non elit consectetur, et tincidunt nisl ultrices. Proin mollis eleifend lacus, ut fermentum justo porta a.


\section{Introduction}
\label{sec:introduction}

Lorem ipsum dolor sit amet, consectetur adipiscing elit. Duis semper hendrerit faucibus. Donec mauris quam, condimentum quis velit et, sodales luctus arcu. Etiam eget rhoncus nunc, in tempus urna. Etiam dignissim quam libero, et aliquam urna tincidunt ac. Aliquam erat volutpat. Aliquam non urna nulla. Aliquam sodales porta tristique. Suspendisse efficitur ante non elit consectetur, et tincidunt nisl ultrices. Proin mollis eleifend lacus, ut fermentum justo porta a.

Lorem ipsum dolor sit amet, consectetur adipiscing elit. Duis semper hendrerit faucibus. Donec mauris quam, condimentum quis velit et, sodales luctus arcu. Etiam eget rhoncus nunc, in tempus urna. Etiam dignissim quam libero, et aliquam urna tincidunt ac. Aliquam erat volutpat. Aliquam non urna nulla. Aliquam sodales porta tristique. Suspendisse efficitur ante non elit consectetur, et tincidunt nisl ultrices. Proin mollis eleifend lacus, ut fermentum justo porta a \cite{smotkin1991dioxygen, bard2012scanning}.

\section{Physical model} \label{sec:reakc_phys}

\subsection{Reaction rate constants}  \label{subs:reakc_const}

\subsubsection{Reaction rate constants}  \label{subs:reakc_const2}

In this research, the kinetic constants for reactions were gathered from references  and adjusted to better fit experimental results (Table \ref{tab:const}). Kinetic constants $k_{-1}$, $k_{-3}$, $k_{-4}$ for reactions were determined from the model and were set to the following values: $k_{-1} = \SI{10}{s^{-1}}$, $k_{-3} = \SI{2000}{M^{-1}s^{-1}}$. The constant $k_{-4}$ was set to zero, because the backward reaction is much slower than other reactions in diffusion-related processes. 

\begin{table}[ht!]
  \centering
  \caption{Kinetic constants and thermodynamic parameters for the GOx catalyzed reaction with $\beta$-D-glucose and oxygen at pH 5.5.}
  \label{tab:const}  
  \vspace{2mm} 
  \def\arraystretch{1.1}
  \begin{tabular}{ | m{8em} | c | c | c | c | c |}
    \hline
    Sugar substrate or thermodynamic parameter & \begin{tabular}{@{}c@{}} $k_{1}$,\\ \si{M^{-1}s^{-1}}\end{tabular} & $k_{2}$, \si{s^{-1}} & \begin{tabular}{@{}c@{}}  $k_{3}$,\\ \si{M^{-1}s^{-1}} \end{tabular} & $k_{4}$, \si{s^{-1}} & ref. \\ \hline
    %Sugar substrate or thermodynamic parameter & $k_{1}$, \si{M^{-1}s^{-1}} & $k_{2}$, \si{s^{-1}} & $k_{3}$, \si{M^{-1}s^{-1}} & $k_{4}$, \si{s^{-1}} & ref. \\ \hline
    $\beta$-D-glucose-1-\ce{^1H} at \SI{25}{\degreeCelsius} & ${\sim}200$ & ${\sim}\num{6000}$ & $\num{1.8d6}$ & $\num{1440}$ & \\ \hline
    $\beta$-D-glucose-1-\ce{^1H} at \SI{25}{\degreeCelsius} & $\num{13158}$ & & $\num{1.8d6}$ &  $\num{1440}$ & \\ \hline
    $\beta$-D-glucose-1-\ce{^1H} at \SI{27}{\degreeCelsius} & $\num{10000}$ & & $\num{2.1d6}$ & $\num{1150}$ & \\ 
\hline
    \midrule
    Used in the model & $\num{3000}$ & $\num{6000}$ & $\num{1.5d6}$ & $\num{1500}$ & \\ [1ex]
    \hline
  \end{tabular}
\end{table}


\section{Mathematical model}  \label{sec:reakc_math}

\begin{figure}[ht!]
\centering
\includegraphics[width=1\linewidth]{chapter_1/Model_domain.png}
\caption{Scheme of simulation domain. All 8 reagents, boundary conditions for $C_{\text{diff}}$ and the direction of outside flux are displayed.}
\label{fig:Domain}
\end{figure}

Measurements of SECM acting in the redox-competition mode are changed into the scheme (\ref{fig:Domain}) due to the radial symmetry around the central axis of the electrode. Radial symmetry is a standard assumption in SECM modelling, though the case of off-centered UME was also investigated.

According to the second Fick’s law , diffusion processes are expressed by the system of partial differential equations (PDE):
\begin{equation}
  \begin{aligned}\label{eq:reakc_eq1}
  \frac{\partial C_{O_2}}{\partial t} &= D_{O_2}\,\Delta C_{O_2},\\
  \frac{\partial C_{Glc}}{\partial t} &= D_{Glc}\,\Delta C_{Glc},\\
  \frac{\partial C_{H_2 O_2}}{\partial t} &= D_{H_2 O_2} \,\Delta C_{H_2 O_2},\\
  \frac{\partial C_{Gll}}{\partial t} &= D_{Gll}\,\Delta C_{Gll},  \quad for\; 0<t\leq T,\; 0<z<d,\; 0<r<r_{glass},
  \end{aligned}
\end{equation}
where:
\begin{itemize}
  \item[] $C_{O_2}$, $C_{Glc}$, $C_{H_2 O_2}$ and $ C_{Gll}$ are concentrations of diffusing reagents and expressed as functions of time $t$ and spatial coordinates $z$ and $r$. Notation $C_{\text{diff}} = C_{\text{diff}} \left( t, z, r \right) = \left( C_{O_2}, C_{Glc}, \allowbreak C_{H_2 O_2}, \allowbreak C_{Gll} \right)$ was used when 4 diffusing re\-agents were considered together.
  \item[] $D_{O_2}$, $D_{Glc}$, $D_{H_2 O_2}$ and $D_{Gll}$ are diffusion coefficients of \ce{O2}, Glc, \ce{H2O2} and Gll.
  \item[] $d$ is the distance between the enzyme-modified surface and the electrode, which is varying from \SIrange{1}{120}{\um} as shown in Fig. \ref{fig:Domain}.
  \item[] $r_{glass} = \SI{80}{\um}$ is the radius of insulated area, $r_{el} = \SI{5}{\um}$ is the radius of electrode.
  \item[] $T$ is the duration of a computational experiment measured in seconds (the evaluation of this parameter is further explained in the next section).
  \item[] The Laplace operator $\Delta$ for concentration function $C$ in cylindrical coordinates with radial symmetry is
  \begin{equation*}
  \Delta C = \frac{1}{r}\frac{\partial }{\partial r} \left( r\frac{\partial C }{\partial r} \right) + \frac{\partial^{2} C}{\partial z^{2}}.
  \end{equation*}
\end{itemize}



\chapter*{General conclusions / Bendrosios išvados }
\label{cha:concl}
\addcontentsline{toc}{chapter}{\MakeUppercase{General conclusions / Bendrosios išvados}} 

It is recommended that these conclusions relate directly to the objectives and defending statements of the thesis. 
The conclusions should accurately reflect the results obtained in the dissertation work.
The recommended number of conclusions is 3-5.

\textbf{Note}. Please do a direct translation of the conclusion to Lithuanian language and put them to the \verb|chapter_concl_lt.tex|.


\textbf{In Lithuanian} Šiame skyriuje pateikite sąrašą pagrindinių disertacijos išvadų. Rekomenduojama, kad šios išvados tiesiogiai sietųsi su darbo uždaviniais ir ginamaisiais teiginiais. Išvadose turėtų tiksliai atsispindėti rezultatai gauti disertacijos darbe. 
Rekomenduojamas išvadų kiekis 3-5.

\textbf{Pataba}.Jeigu disertaciją rašote lietuviškai, tuomet išvadas išverskite į anglų kalbą ir pateikite \verb|chapter_concl.tex| faile, kuris būtų naudojamas angliškoje disertacijos santraukoje.
Jeigu disertaciją rašote angliškai, tuomet išverskite į lietuvių kalbą ir pateikite \verb|chapter_concl_lt.tex| faile.



\backmatter

\restoreParagraph
\pagestyle{plain} 
% [file: bibliography.tex, started: 22-Jun-2005]
%
% PhD Thesis - top level LaTeX source file.
%
% DESCRIPTION
%   This file includes Bibliography chapter of the PhD Thesis. 
%
% CHANGES
%   2005.06.22  *  Started.
%   2008.03.18  *  Adapted to IZ.
%   2020. ...   *  Pagal VU 2017 stand 

%\chapter{Bibliography}
%	\label{chapter:Bibliography}

\chapter*{Bibliography}
\label{cha:bibliography}
\addcontentsline{toc}{chapter}{\MakeUppercase{Bibliography}}
\phantomsection %The \phantomsection command is needed to create a link to a place in the document                      that is not a figure, equation, table, section, subsection, chapter, etc.
	
\parammarks{Bibliography}
	
%\bibliographystyle{plainnat}  %Sitas buvo originaliame sablone, bet galbut patogiau su santrumpomis
\bibliographystyle{abbrvnat} %Bibliografijoje vardai su santrumpomis
\pagestyle{plain}
\bibliography{thesis_bibliography}

%-------------- Baigiasi pagrindine disertacijos dalis  ------------------


\IZParagraph

%\chapter*{Vocabulary - \v{Z}odyn\.elis}
\label{cha:vocabulary}
\addcontentsline{toc}{chapter}{\MakeUppercase{Vocabulary - \v{Z}odyn\.elis}}
%\addcontentsline{toc}{chapter}{\numberline{}Vocabulary - \v{Z}odyn\.elis}
	
\parammarks{Vocabulary - \v{Z}odyn\.elis}

\noindent
 \begin{longtable}[l]{ p{4cm} p{7cm} } %Todal available width is 11cm
    base learner   &  bazinis klasifikatorius \\
    baseline       &  bazinis metodas \\
    change point   &  poky\v{c}io ta\v{s}kas  \\
    concept drift  &  koncepcijos pokytis \\
    context aware  &  kontekstinis  \\
    data mining    &  duomen\k{u} gavyba \\ 
    data source    &  duomen\k{u} \v{s}altinis \\
    gradual drift  &  palaipsnis pokytis  \\
    instance       &  vektorius \\
    instance based learning &  mokymas pagal vektorius \\ 
    label          &  klas\.e \\
    moving average &  slenkantis vidurkis \\ 
    peer methods   &  lyginamieji metodai  \\
    recurring concepts  &  pasikartojantis pokytis (pasikartojan\v{c}ios koncepcijos)  \\
    sequential learning &  mokymas paeiliui  \\
    source         &  \v{s}altinis  \\
    sudden drift   &  staigus pokytis  \\
    supervised learning   &  mokymas su mokytoju  \\
    training window       &  mokymo langas  \\
    unsupervised learning &  mokymasis  \\
    \label{tab:vocabulary}
\end{longtable}

%\begin{compactitem}[]
%\item base learner - bazinis klasifikatorius 
%\item baseline - bazinis metodas 
%\item change point - poky\v{c}io ta\v{s}kas 
%\item concept drift - koncepcijos pokytis
%\item context aware - kontekstinis 
%\item data mining - duomen\k{u} gavyba 
%\item data source - duomen\k{u} \v{s}altinis
%\item gradual drift - palaipsnis pokytis 
%\item instance - vektorius
%\item instance based learning - mokymas pagal vektorius 
%\item label - klas\.e
%\item moving average - slenkantis vidurkis 
%\item peer methods - lyginamieji metodai 
%\item recurring concepts - pasikartojantis pokytis (pasikartojan\v{c}ios koncepcijos)
%\item sequential learning - mokymas paeiliui
%\item source - \v{s}altinis
%\item sudden drift - staigus pokytis
%\item supervised learning - mokymas su mokytoju
%\item training window - mokymo langas 
%\item unsupervised learning - mokymasis 
%\end{compactitem}
  %Jeigu reikia ideti zodyna, pasitikrinti, kur jis turi buti
\appendix  
%\renewcommand{\thesection}{\alph{section}}

\phantomsection
\addcontentsline{toc}{chapter}{Appendix}
\parammarks{Appendix}
	
\chapter*{Appendix}
\label{cha:appendix}

%\index{Algorithms}

\renewcommand{\thefigure}{A.\arabic{figure}}   
\setcounter{figure}{0}
\renewcommand{\thetable}{A.\arabic{table}}
\setcounter{table}{0}
%jeigu yra dar ko nors, pvz algoritmu, teor ir pan, irgi reikia atnaujinti

\section{Algorithms}

In this Appendix we present pseudo codes and the settings used for the peer algorithms, which we implemented and used in experimental evaluation through the thesis. For consistency, the algorithms were named using the first three letters of the surname of the first author.
  %pdfe negraziai atrodo bookmarkai taip pridejus, bet pakenciama

%Change default parameters for Lithuanian language
\setcounter{section}{0}   %šita automatiškai atnaujina \appendix komanda
\renewcommand*{\thesection}{S.\arabic{section}}
\renewcommand{\thechapter}{S}
\renewcommand{\thefigure}{S.\arabic{figure}}   %raidė S įhardcodinta. Galima švelniau pvz su komanda \thechapter
\setcounter{figure}{0}
\renewcommand{\theequation}{S.\arabic{equation}}
\setcounter{equation}{0}
\renewcommand{\thetable}{S.\arabic{table}}
\setcounter{table}{0}
%jeigu yra dar ko nors, pvz algoritmu, teor ir pan, irgi reikia atnaujinti

\lithuanian   %nustatome lietuviu kalba
\sisetup{output-decimal-marker = {,}}  %lietuviski kableliai ir pan
\sisetup{exponent-product=\ensuremath{\cdot}}


\phantomsection
\chapter*{Summary in Lithuanian / Summary in English}
\label{cha:summary_lt}
\addcontentsline{toc}{chapter}{\MakeUppercase{Summary in Lithuanian - Summary in English}}

% NOTES to the author
% For chapter naming, please use APA style https://titlecapitalize.com/title-case-styles/
    % APA, or American Psychological Style, is one of the most commonly used title case styles in academia. It’s mainly used for research papers in social and behavioral sciences. 
    
    % Its title case rules are also easy enough to remember and follow. Capitalize all major words (nouns, pronouns, verbs, adjectives) in your title, as well as prepositions and conjunction with four or more letters. If your title includes a hyphenated word, capitalize both the initial letters before and after the hyphen. Lastly, the first word after a colon or dash is also capitalized when you’re following the APA style of capitalization for your title or headings.

\chapter*{Introduction / Įvadas}
\label{cha:intro_lt}
\addcontentsline{toc}{chapter}{\MakeUppercase{Introduction / Įvadas}} 

The dissertation must be an original scientific work that substantiates \textbf{the research problem, defines the relevance and purpose of the research work, formulates the goal and objectives of the thesis, indicates the novelty of the scientific work, presents defending statements of the thesis}, reviews the research conducted on the topic of the dissertation in the world (abroad and in Lithuania) and their results, presents the applied research methodology (methods), research results discussed, their reliability and relationship with the data of other researchers, conclusions formulated and other important aspects of the dissertation author's opinion. 
An author should talk about bolded items in the introduction.

\textbf{In Lithuanian}. Disertacija turi būti originalus mokslinis darbas, kuriame pagrindžiama \textbf{tiriamoji problema, apibrėžtas darbo aktualumas, tikslas, suformuluoti sprendžiami uždaviniai, nurodytas mokslinio darbo naujumas, ginami disertacijos teiginiai} , apžvelgti disertacijos tema pasaulyje (užsienyje ir Lietuvoje) atlikti tyrimai ir jų rezultatai, pristatyti taikyta tyrimų metodika (metodai), aptarti tyrimų rezultatai, pagrįstas jų patikimumas ir santykis su kitų tyrėjų duomenimis, suformuluotos išvados ir kiti, disertanto nuomone, svarbūs aspektai.
Autorius turi aprašyti paryškintus elementus įvade.


\phantomsection % Removes warning from hyperref package
\section*{Research Area / Tyrimų sritis}
\addcontentsline{toc}{section}{Research Area / Tyrimų sritis} 

In this section, we recommend briefly introducing the readers to the field of research, which is directly related to the research problem and the aim of the dissertation. Present the current situation in this research area, name the important and relevant research conducted in this area, and present their results. You can attach the research problem section to this section.

\textbf{In Lithuanian}.
Šiame skyrelyje rekomenduojame trumpai skaitytojus supažindinti su tyrimų sritimi, kuri tiesiogiai siejasi su su diser\-tacijos problema ir tikslu. Pristatykite, kokia yra dabar situacija šioje srityje, įvardinkite svarbius ir aktualius šioje srityje vykdomus tyrimus ir prisatykite jų rezultatus. Prie šio skyrelio galite prijungti tyrimo problemos skyrelį.

\section*{Research Problem / Tyrimo problema}
\addcontentsline{toc}{section}{Research Problem / Tyrimo problema} 

Definition. A perceived gap between the existing state and a desired state, or a deviation from a norm, standard, or status quo (Bussiness Dictionary\footnote{https://www.bussinessdictionary.com/definition/problem}).

Also:
\begin{itemize}
    \item A problem is a statement that indicates the need to do something.
    \item The problem is a complex, unsolved question. 
    \item A scientific problem can be solved using the steps of the scientific method (e.g., experiments).
\end{itemize}
Before formulating a scientific problem, it is recommended to define the research object and formulate the problem using its concepts.


\textbf{In Lithuanian}. Apibrėžimas. \textbf{Problema} – tai tarpas tarp esamos būsenos ir siekiamos būsenos, arba nukrypimas nuo normos, standarto ar status quo.
Taip pat:
\begin{itemize}
    \item Problema – teiginys, matematikoje ar fizikoje, nurodantis poreikį kažką padaryti.
    \item Problema – sudėtingas neišspręstas klausimas. 
    \item Mokslinė problema – tai klausimas, kuris gali būti atsakytas naudojant mokslinius metodus (pvz., atliekant eksperimentus).
\end{itemize}
Prieš formuojant mokslinę promblemą rekomenduojama apsibrėžti tyrimo objektą ir naudojantis jo sąvokomis suformuluoti problemą.


\section*{Actuality / Darbo aktualumas}
\addcontentsline{toc}{section}{Actuality / Darbo aktualumas} 

The actuality of the research topic is the degree of its importance at the given moment and in this situation for solving these problems, a question or a problem\footnote{\url{https://en.ppt-online.org/424459}}.

In order to show the actuality of the research, you need:
\begin{itemize}
    \item to formulate the research problem
    \item to indicate contradictions found in science or practice that define the research problem
    \item to describe the current situation of the problem with a solution to the problem (is anyone still solving it?)
    \item to describe the importance of research work to society
    \item to generalize and to summarize the best results.
\end{itemize}

\textbf{In Lithuanian}.
Tyrimo temos aktualumas yra sprendžiamos problemos ar klausymo svarbos laipsnis šiuo momentu ir šiuoje situacijoje sprendžiant tyrimo sirties problemas.
Kad parodyti darbo aktualumą reikia:
\begin{itemize}
    \item suformuluoti tyrimo problemą,
    \item nurodyti prieštaravimus, kurie aptinkami moksle arba praktikoje, kurie apibrėžia tyrimo problemą, 
    \item apibūdinti esamą problemos situaciją su problemos sprendimu (ar ją kas nors vis dar sprendžia?),
    \item darbo reikšmingumą visuomenei,
    \item apibendrinti geriausius rezultatus.
\end{itemize}


\section*{Research Object / Tyrimo objektas}
\addcontentsline{toc}{section}{Research Object / Tyrimo objektas} 

\textbf{Research object} or research subject is a thing (e.g. algorithm, person, data type) that is used in your research. The research object connects different research contexts: data, analysis tools, hypotheses, experiments, defending statements, and general conclusions. We recommend identifying from 3 to 5 main research concepts and formulating a coherent sentence from them as a research object.
These concepts should also be used in the thesis title and objective. You can read more about how to identify the object of your research on  \url{http://edutechwiki.unige.ch/en/Methodology_tutorial_-_finding_a_research_subject}.

\textbf{In Lithuanian}. \textbf{Tyrimo objektas} (angl. research object, research subject) – dalykas (pvz. algoritmas, asmuo, duomenų tipas), kuris naudojamas detaliame tyrime. Tyrimo objektas susieja skirtingus tyrimo kontekstus: duomenis, analizės įrankius, hipotezes, eksperimentus, ginamuosius teiginius ir išvadas. Rekomenduojame identifikuoti nuo 3 iki 5 pagrindinių tyrimo konceptų ir iš jų suformuluoti nuoseklų sakinį.
Šie konceptai taip pat turėtų būti panaudoti disertacijos pavadinime ir tiksle. Daugiau apie taip kaip identifikuoti savo tyrimo objektą galima paskaityti \url{http://edutechwiki.unige.ch/en/Methodology_tutorial_-_finding_a_research_subject}.



\section*{Research Aim and Objectives / Tyrimo tikslas ir uždaviniai}
\addcontentsline{toc}{section}{Research Aim and Objectives / Tyrimo tikslas ir uždaviniai} 

The goal of science is an action that is performed to solve a defined scientific problem of your research.
The following types of scientific goals are distinguished\footnote{\url{https://opentext.wsu.edu/carriecuttler/chapter/goals-of-science/}}:
\begin{itemize}
    \item The first and most basic goal of science is \textbf{to describe}.
    \item The second goal of science is \textbf{to predict}.
    \item The third and ultimate goal of science is \textbf{to explain}.
\end{itemize}
When defining the main goal of the dissertation, it is recommended to think about what you managed \textbf{to create} in your dissertation. Thus, defining the goal will make it easier for you to describe the novelty and originality of the research work. 
If your research aims to create a solution, a tool, or an algorithm, it will also be related to scientific aims: to describe, predict, or explain.

For the aim of research, the only requirement is \textbf{it must be measurable}.

Research objectives are smaller actions required to achieve the aim. They must also be measurable. 
The research objective must be relevant, related to the aim, feasible, logical, measurable, and unambiguous.
When executing the objective, we aim to find answers to questions or test research hypotheses.

It is recommended that each objective relates to at least one defending statement and at least one thesis general conclusion. This way you will achieve consistency and integrity of your work. It is recommended that the number of research tasks should be between 3 and 5.

\textbf{In Lithuanian}. 
Mokslo tikslas - tai veiksmas, kuris atliekamas norint išspręsti apibrėžtą savo tyrimo mokslinę problemą. 
Yra išskiriami šie mokslo tikslų tipai:
\begin{itemize}
    \item Pirmas ir bazinis mokslo tikslas – „\textbf{Aprašyti}/apibūdinti/apibrėžti“ (angl. to describe).
    \item Antras mokslo tikslas – „\textbf{Prognozuoti}“ (angl. to predict“).
    \item Trečiasis ir pagrindinis mokslo tikslas „\textbf{Paaiškinti}/Suprasti“ (angl. to explain/understand).
\end{itemize}

Konstruojant pagrindinį disertacijos tikslą rekomenduojama pagalvoti, o ką jūsų darbe pavyko  \textbf{sukurti}, taip apibrėžiant tikslą jums lengviau bus aprašyti darbo naujumą ir orgiginalumą. Darbo tikslas, kaip sukurtas sprendimas, įrankis ar algoritmas siesis ir mokslo tikslais: aprašyti, prognozuoti ar paaiškinti.

Tyrimo tikslui, keliamas vienintelis reikalavimas - \textbf{jis turi būti pamatuojamas}.

Tyrimo uždaviniai tai mažesni veiksmai reikalingi užsibrėžtam tikslui pasiekti. Jie taip pat turi būti  pamatuojami. 
Mokslinis uždavinys turi būti, aktualus, susijęs su tikslu, įvykdomas, logiškas, pamatuojamas, nedviprasmiškas.
Spręsdami uždaviniais mes siekiame surasti atsakymus į klausimus arba patikrinti tyrimo hipotezes.

Rekomenduojama, kad kiekvienas uždavinys sietųsi su bent vienu ginamuoju teiginiu ir su bent su viena disertacijos išvada. Taip pasieksite darbo nuoseklumo ir vientisumo. Rekomenduojama, kad tyrimo uždavinių kiekis būtų nuo  3 iki 5.

\section*{Research Methods / Tyrimo metodai}
\addcontentsline{toc}{section}{Research Methods / Tyrimo metodai} 

Research methods are specific procedures for collecting and analyzing data. Developing your research methods is an integral part of your research design. For more about available research methods, please read on \url{https://www.scribbr.com/category/methodology/}.

\textbf{In Lithuanian}. 
Tyrimo metodai – tai specifinės duomenų rinkimo ir analizės procedūros. Tyrimo metodų kūrimas yra neatsiejama jūsų tyrimo plano dalis. Daugiau tyrimo metodus, skaitykite \url{https://www.scribbr.com/category/methodology/}.


\section*{Scientific Novelty / Mokslinis darbo naujumas} %Scientific Contribution of the Research
\addcontentsline{toc}{section}{Scientific Novelty / Mokslinis naujumas} 

This is one of the main requirements for the dissertation. This means that the dissertation must have a solution to a new scientific problem or a new development that allows expanding the existing boundaries of knowledge in a certain branch of science.

A job is new if:
\begin{itemize}
 \item it is a new interesting question (problem) or topic,
 \item it is a little researched question (problem) or not studied in depth.
\end{itemize}

Novelty can be associated with the reuse of old ideas in new conditions, new fields of science, or practice.

The novelty of the work can be achieved by the following methods:
\begin{itemize}
 \item Introduction of new previously unused sources (methods, laws, theorems) into the existing field of science and their subsequent analysis.
 \item New interpretation of existing and known sources and addition or correction of existing knowledge.
 \item Researching little-understood or analyzed sources of old information and their aspects.
\end{itemize}


\textbf{In Lithuanian}. Tai vienas iš pagrindinių reikalavimui disertacijai. Tai reiškia, kad darbas turi turėti sprendimą naujai mokslinei problemai, arba nauja plėtra kuri leidžia praplėsti egzistuojančias tam tikros mokslo šakos žinių ribas.
Darbas yra naujas, jeigu:
\begin{itemize}
    \item tai naujas įdomus klausimas (problema) ar tema,
    \item tai mažai tyrinėtas klausimas (problema) ar netyrinėtas giliai.
\end{itemize}

Naujumas gali būti susietas su senų idėjų pakartotiniame panaudojimu naujose sąlygose, naujose mokslo ar praktikos srityse.

Darbo naujumas gali būti pasiektas šiais metodais:
\begin{itemize}
    \item Naujų ankščiau nenaudotų šaltinių (metodai, dėsniai, teoremos) įvedimas į esamą mokslo sritį ir vėlesnė jų analizė.
    \item Nauja esamų ir žinomų šaltinių interpretacija ir esamų žinių papildymas ar pataisymas.
    \item Tyrinėjimas mažai suvoktus ar išanalizuotus senos informacijos šaltinius, jų aspektus.
\end{itemize}



\section*{Practical Significance / Praktinė darbo vertė}  %Gali būti impact, kuris žodis geresnis? ar Practical Value of the Research
\addcontentsline{toc}{section}{Practical Significance / Praktinė darbo vertė} 

This section discusses the practical benefits of the results of this dissertation work. Describe where and how they were, are, or can be applied in practice. Or indicate how they contribute to solving practical problems.

\textbf{In Lithuanian}. 
Šiame skyrelyje parašykite apie šio disertacijos darbo rezultatų praktinę naudą. Aprašykite, kur ir kaip jie buvo, yra ar gali būti pritaikyti praktikoje. Arba nurodykite kaip jie prisideda prie praktinių problemų sprendimo.

\section*{Statements to be Defended / Ginamieji teiginiai}
\addcontentsline{toc}{section}{Statements to be Defended / Ginamieji teiginiai}

According to the recommendations set by the Lithuanian Science Council for the structure of the dissertation, the introduction must contain defending statements, which must also be reflected in the conclusions.
There are no more detailed requirements defining what the defending statements are or how many of them there should be - so it is enough to remind you of the previously mentioned advice to read theses defended in a specific discipline and participate in their defenses.

The answer to the question of what are defending statements can be found in another way - by clarifying the concept of "defending statement".
\begin{itemize}
    \item First, it must be a statement (not a question).
    \item Second, it must be a statement that is defended in the thesis, i.e. a statement that is not self-evident and requires justification. 
\end{itemize}
A defending statement is just the equivalent of a thesis with a different name - it states what and how it is intended to be proved.
In many cases, the equivalent of a defending statement can be a hypothesis - in the dissertation, we test assumptions about an expected relationship.

The recommended number of defending statements is from 3 to 5. They must also be related to the research objectives and conclusions.

\textbf{In Lithuanian}. 
Pagal  Lietuvos  mokslo  tarybos  nustatytas  rekomendacijas  disertacijos struktūrai, įvade turi būti suformuluoti ginamieji teiginiai, kurie taip pat jie turi atsispindėti išvadose. 
Detalesnių reikalavimų, apibrėžiančių, kas  yra  ginamieji  teiginiai  ar kiek jų  turėtų būti,  nėra nustatyta – todėl telieka priminti jau anksčiau minėtą patarimą skaityti konkrečioje disciplinoje gintas disertacijas ir dalyvauti jų gynimuose (Doktorantūros studijų kokybės valdymas, metodinė medžiaga doktorantų vadovams ir doktorantams, VU, 2013 m.\footnote{\url{https://www.mii.vu.lt/files/doc/lt/doktorantura/dokumentu_sablonai/doktoranturos_studiju_kokybes_valdymas_metodine_medziaga_doktorantu_vadovams_ir_doktorantams1.pdf}
}).

Atsakymo į klausimą, kas yra ginamieji teiginiai, galima ieškoti ir kitu keliu – išsiaiškinant „ginamojo teiginio“ sąvoką. 
\begin{itemize}
    \item Pirma, tai turi būti teiginys (o ne klausimas). 
    \item Antra, tai turi būti teiginys, kuris disertacijoje yra ginamas, t. y. teiginys, kuris nėra savaime akivaizdus ir kurį reikia pagrįsti. 
\end{itemize}
Ginamasis teiginys yra tik kitaip įvardintas tezės ekvivalentas – juo įvardijama, kas ir kaip ketinama įrodinėti. 
Ginamojo teiginio ekvivalentu daugeliu atveju gali būti hipotezės – disertacijoje tikrinami spėjimai apie numatomą priežastinį ryšį.

Rekomenduojamas ginamųjų teiginių kiekis nuo  3 iki 5. Ir jie turi sietis su turimo uždaviniais ir darbo išvadomis.


\section*{Approbation and Publications of the Research/ Tyrimo aprobavimas ir publikavimas} %Darbo rezultatų aprobavimas
\addcontentsline{toc}{section}{Approbation and Publications of the Research/ Tyrimo aprobavimas ir publikavimas} 

Your dissertation must meet the formal requirements, i.e., the results presented in the dissertation had to be published in at least two articles in international research journals with a citation index in the Clarivate Analytics Web of Science (CA WoS) database and discussed at two international conferences.
In this section, a list of publications published in the CA WoS database (same as \nameref{cha:publications}) and a list of all other publications should be presented. Additionally, provide a list of all scientific events where you have given presentations on the topic of the dissertation.

\textbf{In Lithuanian}. 
Jūsų disertacija turi tenkinti formalius reikalavimus, t.y. disertacijoje pristatomi rezultatai turėjo būti išspausdinti bent dviejuose žurnaluose indeksuojamuose CA WoS duomenų bazėje, ir aptarti bend dvejose tarptautinėse konferencijose.
Šiame skyrelyje pateikite sąrašą visų disetacijos tema publikuotų darbų iš jų išskiriant publikacijas pakslebtas Web Of Science duomenų bazėje (tas pats kaip ir skyriuje \nameref{cha:publications}). Papildomai pateikite sąrašą visų mokslinių renginių, kuriuose skaitėte pranešimus disertacijos tema.


\section*{Outline of the Thesis/ Disertacijos strukūra}
\addcontentsline{toc}{section}{Outline of the Thesis/ Disertacijos strukūra} 

In this section, present the main sections of the thesis and the overall size of the thesis.

This doctoral thesis consists of an introduction, ... chapters, conclusions, and a summary in the Lithuanian language. The introduction section provides an introduction to the research and an overview of the dissertation. The first chapter is ...

... bibliographic references are included at the end of the thesis. The disertation consist of ... pages, ... figures and ... tables.


\textbf{In Lithuanian}. 

Šiame skyrelyje pristatykite pagdindinius disertacijos skyrius ir bendrą disertacijos apimtį.



\textbf{Note} Please translate all parts of the introduction to Lithuanian and put them into the \verb|chapter_intro_lt.tex| file.
\textbf{In Lithuanian}. \textbf{Pastaba}Angliško ir lietuviško įvado turinys turi būti identiškas turinio prasme. Išverstą tekstą patalpinkite \verb|chapter_intro_lt.tex| failą.

% !TEX encoding = UTF-8 Unicode

\pagestyle{plain}

%\lithuanian

\phantomsection % Removes warning form hyperref package
\section*{Tyrimų sritis}
\addcontentsline{toc}{section}{Research Area / Tyrimų sritis}

Pasitelkus kompiuterinį modeliavimą disertacijoje tiriamos su\-dė\-tin\-gos cheminės ir biofizikinės sistemos, kurios yra aprašomos dalinių išvestinių lygtimis (DIL) su netiesinėmis kraštinėmis sąlygomis ir DIL sudėtingos geometrijos (nestačiakampėse) srityse. Šios DIL sprendžiamos baigtinių skirtumų metodu ir kitais skaitiniais algoritmais. 

%Disertacijoje tiriamos sudėtingos cheminės ir biofizikinės sistemos naudojant kompiuterinį modeliavimą. Šios sistemos yra aprašomos dalinių išvestinių lygtimis (DIL) su netiesinėmis kraštinėmis sąlygomis ir DIL sudėtingos geometrijos (nestačiakampėse) srityse. Šios DIL sprendžiamos baigtinių skirtumų metodu ir kitais skaitiniais algoritmais.  
%The study is focused on computer modelling of complex chemical and biophysical systems, which are described by partial differential equations (PDEs) with nonlinear boundary conditions and PDEs in various complex (non-rectangular) domains. Differential problems are solved using numerical methods
%Tyrimas skirtas išspręsti dalinių išvestinių lygtims (DIL) su netiesinėmis kraštinėmis sąlygomis sistemas ir išspręsti DIL sudėtingos geometrijos (nestačiakampėse) srityse, naudojant skaitinius metodus.

DIL sprendimo su netiesinėmis kraštinėmis sąlygomis problemos kyla dėl cheminių ir biologinių procesų matematinio modeliavimo. 
Tyrimai nestačiakampėse srityse yra aktualūs dėl poreikio įvertinti matavimo prietaisų paklaidas, atsirandančias dėl geometrijos nukrypimo nuo standarto.
%Tyrimai nestačiakampėse srityse yra aktualūs dėl poreikio modeliuoti nukrypimus nuo normos įrangoje, naudojamoje cheminiams ir biologiniams eksperimentams. 
DIL netiesinės sistemos yra pritaikytos tyrinėti chemoterapinių vaistų patekimą į audinius.





\section*{Tikslas}
\addcontentsline{toc}{section}{Tikslas} 
...





%--------------- SECM Redox reakciju skyrius ---------------------------------
%--------------------------------------------------------------------------

\section{SECM modeliavimas oksidacijos-redukcijos konkurencijos režime}
\label{sec:santr_reakc}



\subsection{Matematinis modelis}

Dėl simetrijos aplink centrinę elektrodo ašį modelis užrašomas cilindrinėse koordinatėse. Cilindro formos srityje atliekami SECM matavimai yra pakeisti į 2D sritį \ref{fig:santr_Domain} pav.


\begin{figure}[ht!]
\centering
\includegraphics[width=0.8\linewidth]{summary/Model_domainLT.png}
\caption{Modeliavimo srities schema. Pavaizduotos $8$ modeliuotos medžiagos - $4$ difunduojantys reagentai bei $4$ fermento GOx formos, kraštinės sąlygos 4-ioms difunduojančioms medžiagoms ir išorinis srautas.}
\label{fig:santr_Domain}
\end{figure}

Difuzijos procesai išreiškiami antruoju Fiko dėsniu:
\begin{equation}
  \begin{aligned}\label{eq:santr_eq1}
  \frac{\partial C_{O_2}}{\partial t} &= D_{O_2}\,\Delta C_{O_2},\\
  \frac{\partial C_{Glc}}{\partial t} &= D_{Glc}\,\Delta C_{Glc},\\
  \frac{\partial C_{H_2 O_2}}{\partial t} &= D_{H_2 O_2} \,\Delta C_{H_2 O_2},\\
  \frac{\partial C_{Gll}}{\partial t} &= D_{Gll}\,\Delta C_{Gll},  \quad 0<t\leq T,\; 0<z<d,\; 0<r<r_{glass}.
  \end{aligned}
\end{equation}
Šiose lygtyse:
\begin{itemize}
  \item[] $C_{O_2}$, $C_{Glc}$, $C_{H_2 O_2}$ ir $C_{Gll}$ yra atitinkamų difunduojančių re\-a\-gen\-tų koncentracijos, kurios išreiškiamos kaip laiko $t$, er\-dvi\-nių ko\-or\-di\-na\-čių $z$ ir $r$ funkcijos. 
  \item[] $D_{O_2}$, $D_{Glc}$, $D_{H_2 O_2}$ ir $D_{Gll}$ yra difuzijos koeficientai.
  \item[] $d$ yra atstumas tarp fermentu modifikuoto paviršiaus ir elektrodo. Skaitinio eksperimento metu $d$ keičiamas nuo $\SI{1}{\um}$ iki $\SI{120}{\um}$. Tai atitinka elektrodo stumdymą aukštyn ir žemyn cheminio eksperimento metu.
  \item[] $r_{glass} = \SI{80}{\um}$ yra izoliuotos srities spindulys.
  \item[] $T$ yra skaičiavimo eksperimento trukmė, matuojama sekundėmis.
  \item[] Laplaso operatorius $\Delta$ cilindrinėse koordinatėse su centrine simetrija yra
  \begin{equation*}
  \Delta C = \frac{1}{r}\frac{\partial C }{\partial r} \left( r\frac{\partial C }{\partial r} \right) + \frac{\partial^{2} C}{\partial z^{2}}.
  \end{equation*}
\end{itemize}


\chapter*{General conclusions / Bendrosios išvados }
\label{cha:concl_lt}
\addcontentsline{toc}{chapter}{\MakeUppercase{General conclusions - Bendrosios išvados}} 


...


\british  %griztame prie anglu kalbos
%%%%%%%%%%%%%%%%%%%%%%%%%%%%%%%%%%%%%%%%%%%%%%%%%%%%%%%%%%%%%%%%%%%%%%%%%%%%%%%%%%%%%%%%%%%%%%
%% CV is not required in the dissertation body, but it is required in the dissertation summary
%% CV neprašo įdėti į disertacijos turinį, bet prašo įdeti po santraukos
\chapter*{Curriculum Vitae - Gyvenimo aprašymas}
\label{cha:cv}
\addcontentsline{toc}{chapter}{\MakeUppercase{Curriculum Vitae - Gyvenimo aprašymas}}
% \parammarks{Curriculum Vitae}


Vardas Pavardė graduated from ... 

The author's curriculum vitae should be at max one paragraph long.    %Nors cv reikalavimo nera, VU doktoranturos skyrius turetu paprasyti prideti po santraukos gyvenimo aprasyma

\backmatter
\IZParagraph 
%%%%%%%%%%%%%%%%%%%%%%%%%%%%%%%%%%%%%%%%%%%%%%%%%%%%%%%%%%%%%%%%%%%%%%%%%%%%%%%%%%%%%%%%%%%%%%
%% Acknowledgements / Padėka

\chapter*{Acknowledgements - Padėka}
\label{cha:acknowledgements}
\addcontentsline{toc}{chapter}{\normalfont\MakeUppercase{Acknowledgements - Padėka}}

\phantomsection %The \phantomsection command is needed to create a link to a place in the document that is not a figure, equation, table, section, subsection, chapter, etc.

%%%%%%%%%%%%%%%%%%%%%%%%%%%%%%%%%%%%%%%%%%%%%%%%%%%%%%%%%%%%%%%%%%%%%%%%%%%%%%%%%%%%%%%%%%%%%%
%%  Texts of the acknowledgements / Padėkos tekstas

Your acknowledgment text goes here. Try to keep it within one page.

It is not required / Padėka yra neprivaloma

% Use as it is necessary
% {\flushright  
%     \thesisAuthorName \ \thesisAuthorSurname \\ Vilnius\\ \today\\ 
% }
  %Neprivalomas

\chapter*{List of author publications - Disertacijos autoriaus publikacijų sąrašas}
\label{cha:publications} 
\addcontentsline{toc}{chapter}{\MakeUppercase{List of author publications - Disertacijos autoriaus publikacijų sąrašas}}	
%\parammarks{\chapter*{List of publications - Publikacijų sąrašas}}

%Čia galima įdėti ir visą publikacijų sąrašą, jei norite
%\input{publications/author's_publications.bib}

%\index{publications} 

%Bibliografinio aprašo elementų tvarka
%Straipsnis: 
%1. pirminė atsakomybė (autoriaus vardas ir pavardė) (privaloma); 2. straipsnio antraštė ir paantraštė (privaloma); 3. žurnalo pavadinimas (privaloma); 4. žurnalo numeracija (tomas, numeris, metai) (privaloma); 5. paginacija, puslapių intervalas; 6. elektroninio straipsnio identifikatoriai arba interneto nuoroda: a) DOI numeris; b) interneto nuoroda (jeigu nėra DOI numerio).
%Knyga: 
%1. pirminė atsakomybė (autoriaus(ių) pavardė ir vardas; kolektyvo/organizacijos pavadinimas, jei tai kolektyvinis autorius) (privaloma); 2. knygos antraštė ir paantraštė (privaloma); 3. antrinė atsakomybė (vertėjai, redaktoriai, sudarytojai) (jeigu nurodyti antraštiniame lape); 4. leidimas (jeigu ne pirmas leidimas); 5. tomas (jeigu cituojamas daugiatomis leidinys); 6. serijos pavadinimas (neprivaloma); 7. skelbimo informacija (leidimo vieta, leidykla, metai) (privaloma); 8. elektroninės knygos identifikatoriai arba interneto nuoroda: a) DOI numeris; b) interneto nuoroda (jeigu nėra doi numerio).

\makeatletter
\newcommand{\authorpaperlabel}[2]{%
    \@bsphack
    \begingroup
        \def\label@name{#1}%
        \label@hook
        \protected@write\@auxout{}{
            \string\newlabel{#1}{%
                {#2} %current label
                {\thepage} %
                {\@currentlabelname}%
                {\@currentHref}%
                {}%
            } %
        }%
    \endgroup
    \@esphack
}%
\makeatother
  
\newcounter{itemnumber}
\newenvironment{publicationlist}[1]{% Prefix
\setcounter{itemnumber}{0}%
\begin{list}{\textbf{#1}}{}%
}{\end{list}}

\newcommand{\publicationentry}[2]{% Number, Text
\stepcounter{itemnumber}%
\item {[}#1.\theitemnumber{]}\ #2 \authorpaperlabel{mypaper:#1.\theitemnumber} {[#1.\theitemnumber]}%
\label{[#1.\theitemnumber]}%
}

\textbf{Articles/Straipsniai:}

\begin{publicationlist}{}
    \publicationentry{A}{Vardas, P. \& Kitas, A. Title of the book - knygos pavadinimas. (CRC Press,2024)}
    \publicationentry{A}{lorem ipsum dwa}
    \publicationentry{A}{lorem ipsum tri}
\end{publicationlist}

\textbf{Books/Knygos:}
\begin{publicationlist}{}
    \publicationentry{B}{ardas, P. \& Kitas, A. Title of the paper - straipsnio pavadinimas. {\em Biochimica Et Biophysica Acta (BBA)-Biomembranes}. \textbf{1061}, 33-38 (2024), https://www.mii.lt}
    \publicationentry{B}{lorem ipsum dwa}
    \publicationentry{B}{lorem ipsum tri}
\end{publicationlist}

\ref{mypaper:A.1}
\ref{mypaper:A.3}
\ref{mypaper:B.2}   %Publikaciju sarasas ir ju kopijos

%\restoreParagraph
%%%%%%%%%%%%%%%%%%%%%%%%%%%%%%%%%%%%%%%%%%%%%%%%%%%%%%%%%%%%%%%%%%%%%%%%%%%%%%%%%%%%%%%%%%%%%%%
%% [file: index.tex, started: 25-Aug-2005]
%%
%% PhD Thesis - top level LaTeX source file.
%%
%% DESCRIPTION
%%   This file includes Index chapter of the PhD Thesis. 
%%   USe command \index{keyword} to add "keyword" to the index
%%
%% CHANGES
%%   2005.08.25  *  Started.
%%   2008.03.18  *  Adapted to IZ.
%%   2024.08.30  *  Fixed to fit VU dissertation template

\chapter{Index}
\label{chapter:Index}
% \addtocontents{toc}{\protect\enlargethispage{2\baselineskip}}
\phantomsection
% \addcontentsline{toc}{chapter}{\numberline{}Index} % \numberline{} atitraukia į šalį
\addcontentsline{toc}{chapter}{\MakeUppercase{Index}}

%% Show index items in two-column style sorted by name
%To add something to the index please use command \index{keyword}
{\raggedright
    \printindex
}
  %indexas - nuorodos i puslapius, kur panaudota kokia svarbi savoka. Nereikalinga	

\pagestyle{empty}

\iffalse   
\begin{center}   %Jeigu truktu puslapiu. Leidykla prasys kad dalintusi is 4
NOTES
\end{center}
\newpage
\fi

\cleardoublepage



\vspace{165mm}

\begin{flushleft}
Rokas Astrauskas
	
Computer Modelling of Reaction-Diffusion Processes\\ 
in Scanning Electrochemical Microscopy and in Cell Spheroids

Doctoral Dissertation\\
Natural Sciences\\
Informatics (N 009)\\
Thesis Editor: ...


\vspace{3cm}

Reakcijos-difuzijos procesų kompiuterinis modeliavimas\\
skenuojančioje mikroskopijoje ir ląstelių sferoiduose

Daktaro disertacija\\
Gamtos mokslai\\
Informatika (N 009)\\
Santraukos redaktorė: ...


\end{flushleft}




\vspace{165mm}

{
	\vspace*{\fill}
	
	\centering
Vilnius University Press\\
9 Saulėtekio Ave., Building III, LT-10222 Vilnius\\
Email: info@leidykla.vu.lt, www.leidykla.vu.lt\\
Print run of ... copies\\
}
  %pacioje pabaigoje - darsyk pavadinimas ir leidykla

\end{document}