%%%%%%%%%%%%%%%%%%%%%%%%%%%%%%%%%%%%%%%%%%%%%%%%%%%%%%%%%%%%%%%%%%%%%%%%%%%%%%%%%%%%%%%%%%%%%%
%% Chapter 1

%% Set chapters counter to zero.
\setcounter{chapter}{0}

\chapter{\MakeUppercase{Literature review / Literatūros apžvalga}} %Make it upper case to appear in TOC correctly
\label{cha:review}

%%%%%%%%%%%%%%%%%%%%%%%%%%%%%%%%%%%%%%%%%%%%%%%%%%%%%%%%%%%%%%%%%%%%%%%%%%%%%%%%%%%%%%%%%%%%%%
%%  The Literature review content starts here./ Literatūros apžvalgos tekstas prasideda čia

A literature review surveys prior research published in books, scholarly articles, and any other sources relevant to a particular issue, area of research, or theory, and by so doing, provides a description, summary, and critical evaluation of these works in relation to the research problem being investigated. Literature reviews are designed to provide an overview of sources you have used in researching a particular topic and to demonstrate to your readers how your research fits within existing scholarship about the topic.  (Fink, Arlene. Conducting Research Literature Reviews: From the Internet to Paper. Fourth edition. Thousand Oaks, CA: SAGE, 2014.)\footnote{https://libguides.usc.edu/writingguide/literaturereview}. 

You should rename the title of this chapter to fit your dissertation domain. 

It must describe the research conducted on the topic of the dissertation in Lithuania and abroad and show the contribution of the author of the dissertation to the issue under consideration.
Although there are no formal requirements for the review, it is recommended that it be no longer than $50\%$ of the entire text of the dissertation (excluding the introduction). The recommended volume of the literature review is $33\%$ of the entire dissertation text.

\textbf{Note}. Start each chapter of the thesis with a brief introduction of what the chapter will be about and indicate the author's work it is based on, citing them properly (e.g., \ref{mypaper:A.1}). At the end of the chapter, it is recommended to provide a chapter conclusions section, where you can find a list of conclusions that the author has obtained in this section.


\textbf{In Lithuanian}. Literatūros apžvalga, tai apžvalga ir tyrimas: knygų, mokslinių straipsnių ir kitų susijusių su pasirinkta tyrimo problema šalinių, tyrimo srities, teorijos. 
Literatūros apžvalga leidžia apibrėžti, apibendrinti ir kritiškai įvertinti darbus susijusius su norima tirti tyrimo problema.
Literatūros apžvalga leidžia pademonstruoti skaitytojams kad jūsų tyrimas priklauso platesniam tyrimų ratui, sričiai. (Fink, Arlene. Conducting Research Literature Reviews: From the Internet to Paper. Fourth edition. Thousand Oaks, CA: SAGE, 2014.) 


Šio skyriaus pavadinimas parenkamas taip, kad atitiktų apžvalgą disertacijos srityje.

Tyrimų apžvalga. Joje turi būti aprašyti disertacijos tema \textbf{Lietuvoje ir užsienyje atlikti tyrimai} ir parodyta, koks yra disertacijos autoriaus indėlis į nagrinėjamą problematiką.

Nors formalių reikalavimų apžvalgai nėra, tačiau rekomenduojama kad ji būtų ne ilgesnė negu $50\ \%$ viso disertacijos teksto (atmetus įvadą). Rekomenduojama literatūros apžvalgos apimtis $33\ \%$ viso disertacijos teksto.

\textbf{Pastaba}. Kiekvieną disertacijos skyrių pradėkite trumpu įvadu apie ką bus šis skyrius ir nurodykute kokių autoriaus darbų pagrindu jisai parašytas, juos tinkamai pacituojant (pvz., \ref{mypaper:A.1}). Skyriaus pabaigoje rekomenduojama pateikti skyriaus išvadų skyrelį, kuriame paeikti sąrašą išvadų, kokios autoriaus buvo gautos šiame skyrelyje.


\section{Conclusions of the Chapter / Skyriaus išvados}
It is preferable that each chapter ends with chapter conclusions, where the author briefly presents the main results obtained and presented in this chapter.

\textbf{In Lithuanian}. Pageidautina, kad kiekvienas skyrius baigtųsi skyriaus išvadomis, kuriose autorius trumpai pristatytų pagrindinius rezultatus gautus ir pristatytus šiame skyriuje.
