% \titlespacing*{\chapeter} {0pt}{3.5ex plus 1ex minus .2ex}{2.3ex plus .2ex}
% \titlespacing*{\section} {0pt}{3.5ex plus 1ex minus .2ex}{2.3ex plus .2ex}
% \titlespacing*{\subsection} {0pt}{3.25ex plus 1ex minus .2ex}{1.5ex plus .2ex}
% \titlespacing*{\subsubsection}{0pt}{3.25ex plus 1ex minus .2ex}{1.5ex plus .2ex}
% \titlespacing*{\paragraph} {0pt}{3.25ex plus 1ex minus .2ex}{1em}

\def \mychapter {
\titleformat{\chapter}[display]
%{chapter style}{chapter text} {space before chapter title} {chapter title format}
{\normalfont}{-}{-10mm}{\thechapter . \centering\normalfont\MakeUppercase}
} 



\setcounter{chapter}{0}
\chapter{\MakeUppercase{Literature review / Literatūros apžvalga}} %Make it upper case to appear in TOC correctly
\label{cha:review}
% You should rename the title of this chapter to fit your dissertation domain
% Skyriaus pavadinimas parenkamas taip, kad atitiktų apžvalgą disertacijos srityje

Lorem ipsum dolor sit amet, consectetur adipiscing elit. Duis semper hendrerit faucibus. Donec mauris quam, condimentum quis velit et, sodales luctus arcu. Etiam eget rhoncus nunc, in tempus urna. Etiam dignissim quam libero, et aliquam urna tincidunt ac. Aliquam erat volutpat. Aliquam non urna nulla. Aliquam sodales porta tristique. Suspendisse efficitur ante non elit consectetur, et tincidunt nisl ultrices. Proin mollis eleifend lacus, ut fermentum justo porta a.


\section{Introduction}
\label{sec:introduction}

Lorem ipsum dolor sit amet, consectetur adipiscing elit. Duis semper hendrerit faucibus. Donec mauris quam, condimentum quis velit et, sodales luctus arcu. Etiam eget rhoncus nunc, in tempus urna. Etiam dignissim quam libero, et aliquam urna tincidunt ac. Aliquam erat volutpat. Aliquam non urna nulla. Aliquam sodales porta tristique. Suspendisse efficitur ante non elit consectetur, et tincidunt nisl ultrices. Proin mollis eleifend lacus, ut fermentum justo porta a.

Lorem ipsum dolor sit amet, consectetur adipiscing elit. Duis semper hendrerit faucibus. Donec mauris quam, condimentum quis velit et, sodales luctus arcu. Etiam eget rhoncus nunc, in tempus urna. Etiam dignissim quam libero, et aliquam urna tincidunt ac. Aliquam erat volutpat. Aliquam non urna nulla. Aliquam sodales porta tristique. Suspendisse efficitur ante non elit consectetur, et tincidunt nisl ultrices. Proin mollis eleifend lacus, ut fermentum justo porta a \cite{smotkin1991dioxygen, bard2012scanning}.

\section{Physical model} \label{sec:reakc_phys}

\subsection{Reaction rate constants}  \label{subs:reakc_const}

\subsubsection{Reaction rate constants}  \label{subs:reakc_const2}

In this research, the kinetic constants for reactions were gathered from references  and adjusted to better fit experimental results (Table \ref{tab:const}). Kinetic constants $k_{-1}$, $k_{-3}$, $k_{-4}$ for reactions were determined from the model and were set to the following values: $k_{-1} = \SI{10}{s^{-1}}$, $k_{-3} = \SI{2000}{M^{-1}s^{-1}}$. The constant $k_{-4}$ was set to zero, because the backward reaction is much slower than other reactions in diffusion-related processes. 

\begin{table}[ht!]
  \centering
  \caption{Kinetic constants and thermodynamic parameters for the GOx catalyzed reaction with $\beta$-D-glucose and oxygen at pH 5.5.}
  \label{tab:const}  
  \vspace{2mm} 
  \def\arraystretch{1.1}
  \begin{tabular}{ | m{8em} | c | c | c | c | c |}
    \hline
    Sugar substrate or thermodynamic parameter & \begin{tabular}{@{}c@{}} $k_{1}$,\\ \si{M^{-1}s^{-1}}\end{tabular} & $k_{2}$, \si{s^{-1}} & \begin{tabular}{@{}c@{}}  $k_{3}$,\\ \si{M^{-1}s^{-1}} \end{tabular} & $k_{4}$, \si{s^{-1}} & ref. \\ \hline
    %Sugar substrate or thermodynamic parameter & $k_{1}$, \si{M^{-1}s^{-1}} & $k_{2}$, \si{s^{-1}} & $k_{3}$, \si{M^{-1}s^{-1}} & $k_{4}$, \si{s^{-1}} & ref. \\ \hline
    $\beta$-D-glucose-1-\ce{^1H} at \SI{25}{\degreeCelsius} & ${\sim}200$ & ${\sim}\num{6000}$ & $\num{1.8d6}$ & $\num{1440}$ & \\ \hline
    $\beta$-D-glucose-1-\ce{^1H} at \SI{25}{\degreeCelsius} & $\num{13158}$ & & $\num{1.8d6}$ &  $\num{1440}$ & \\ \hline
    $\beta$-D-glucose-1-\ce{^1H} at \SI{27}{\degreeCelsius} & $\num{10000}$ & & $\num{2.1d6}$ & $\num{1150}$ & \\ 
\hline
    \midrule
    Used in the model & $\num{3000}$ & $\num{6000}$ & $\num{1.5d6}$ & $\num{1500}$ & \\ [1ex]
    \hline
  \end{tabular}
\end{table}


\section{Mathematical model}  \label{sec:reakc_math}

\begin{figure}[ht!]
\centering
\includegraphics[width=1\linewidth]{chapter_1/Model_domain.png}
\caption{Scheme of simulation domain. All 8 reagents, boundary conditions for $C_{\text{diff}}$ and the direction of outside flux are displayed.}
\label{fig:Domain}
\end{figure}

Measurements of SECM acting in the redox-competition mode are changed into the scheme (\ref{fig:Domain}) due to the radial symmetry around the central axis of the electrode. Radial symmetry is a standard assumption in SECM modelling, though the case of off-centered UME was also investigated.

According to the second Fick’s law , diffusion processes are expressed by the system of partial differential equations (PDE):
\begin{equation}
  \begin{aligned}\label{eq:reakc_eq1}
  \frac{\partial C_{O_2}}{\partial t} &= D_{O_2}\,\Delta C_{O_2},\\
  \frac{\partial C_{Glc}}{\partial t} &= D_{Glc}\,\Delta C_{Glc},\\
  \frac{\partial C_{H_2 O_2}}{\partial t} &= D_{H_2 O_2} \,\Delta C_{H_2 O_2},\\
  \frac{\partial C_{Gll}}{\partial t} &= D_{Gll}\,\Delta C_{Gll},  \quad for\; 0<t\leq T,\; 0<z<d,\; 0<r<r_{glass},
  \end{aligned}
\end{equation}
where:
\begin{itemize}
  \item[] $C_{O_2}$, $C_{Glc}$, $C_{H_2 O_2}$ and $ C_{Gll}$ are concentrations of diffusing reagents and expressed as functions of time $t$ and spatial coordinates $z$ and $r$. Notation $C_{\text{diff}} = C_{\text{diff}} \left( t, z, r \right) = \left( C_{O_2}, C_{Glc}, \allowbreak C_{H_2 O_2}, \allowbreak C_{Gll} \right)$ was used when 4 diffusing re\-agents were considered together.
  \item[] $D_{O_2}$, $D_{Glc}$, $D_{H_2 O_2}$ and $D_{Gll}$ are diffusion coefficients of \ce{O2}, Glc, \ce{H2O2} and Gll.
  \item[] $d$ is the distance between the enzyme-modified surface and the electrode, which is varying from \SIrange{1}{120}{\um} as shown in Fig. \ref{fig:Domain}.
  \item[] $r_{glass} = \SI{80}{\um}$ is the radius of insulated area, $r_{el} = \SI{5}{\um}$ is the radius of electrode.
  \item[] $T$ is the duration of a computational experiment measured in seconds (the evaluation of this parameter is further explained in the next section).
  \item[] The Laplace operator $\Delta$ for concentration function $C$ in cylindrical coordinates with radial symmetry is
  \begin{equation*}
  \Delta C = \frac{1}{r}\frac{\partial }{\partial r} \left( r\frac{\partial C }{\partial r} \right) + \frac{\partial^{2} C}{\partial z^{2}}.
  \end{equation*}
\end{itemize}

