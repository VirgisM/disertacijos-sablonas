%%%%%%%%%%%%%%%%%%%%%%%%%%%%%%%%%%%%%%%%%%%%%%%%%%%%%%%%%%%%%%%%%%%%%%%%%%%%%%%%%%%%%%%%%%%%%%
%% A glossary is a specialized list of terms and definitions often found at the end of a book, article, or in a field of study. It %% is an alphabetical list of words pertaining to a specific subject, with explanations; a reference tool to aid understanding of %%specialized language. 

\chapter*{Glossary - \v{Z}odyn\.elis}
\label{cha:glossary}
\addcontentsline{toc}{chapter}{\MakeUppercase{Glossary - \v{Z}odyn\.elis}}

\noindent
 \begin{longtable}[l]{ p{4cm} p{7cm} } %Todal available width is 11cm
    base learner   &  bazinis klasifikatorius \\
    baseline       &  bazinis metodas \\
    change point   &  poky\v{c}io ta\v{s}kas  \\
    concept drift  &  koncepcijos pokytis \\
    context aware  &  kontekstinis  \\
    data mining    &  duomen\k{u} gavyba \\ 
    data source    &  duomen\k{u} \v{s}altinis \\
    gradual drift  &  palaipsnis pokytis  \\
    instance       &  vektorius \\
    instance based learning &  mokymas pagal vektorius \\ 
    label          &  klas\.e \\
    moving average &  slenkantis vidurkis \\ 
    peer methods   &  lyginamieji metodai  \\
    recurring concepts  &  pasikartojantis pokytis (pasikartojan\v{c}ios koncepcijos)  \\
    sequential learning &  mokymas paeiliui  \\
    source         &  \v{s}altinis  \\
    sudden drift   &  staigus pokytis  \\
    supervised learning   &  mokymas su mokytoju  \\
    training window       &  mokymo langas  \\
    unsupervised learning &  mokymasis  \\
    \label{tab:vocabulary}
\end{longtable}

%Old TODO delete
%\begin{compactitem}[]
%\item base learner - bazinis klasifikatorius 
%\item baseline - bazinis metodas 
%\item change point - poky\v{c}io ta\v{s}kas 
%\item concept drift - koncepcijos pokytis
%\item context aware - kontekstinis 
%\item data mining - duomen\k{u} gavyba 
%\item data source - duomen\k{u} \v{s}altinis
%\item gradual drift - palaipsnis pokytis 
%\item instance - vektorius
%\item instance based learning - mokymas pagal vektorius 
%\item label - klas\.e
%\item moving average - slenkantis vidurkis 
%\item peer methods - lyginamieji metodai 
%\item recurring concepts - pasikartojantis pokytis (pasikartojan\v{c}ios koncepcijos)
%\item sequential learning - mokymas paeiliui
%\item source - \v{s}altinis
%\item sudden drift - staigus pokytis
%\item supervised learning - mokymas su mokytoju
%\item training window - mokymo langas 
%\item unsupervised learning - mokymasis 
%\end{compactitem}

% \newglossary[tlg]{subs}{tld}{tdn}{Subscripts and superscripts}