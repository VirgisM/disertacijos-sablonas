%%%%%%%%%%%%%%%%%%%%%%%%%%%%%%%%%%%%%%%%%%%%%%%%%%%%%%%%%%%%%%%%%%%%%%%%%%%%%%%%%%%%%%%%
%% Preamble file is used to define pdf parameters, to include required packages

\usepackage{ifpdf}

\def \thesisAuthorName {Vardas}
\def \thesisAuthorSurname {Pavardė}

%DOI numeris (suteikiamas atsiuntus disertaciją spausdinti)
\def \thesisDOI {https://doi.org/}

%Jeigu ORCID dar neturite, nemokamai galite susikurti https://orcid.org/)
\def \thesisORCID {https://orcid.org/0000-0001-2345-6789} 

\def \thesisYear {2024}
\def \thesisPreparationStartYear {2020}

%Įveskite daktaro disertacijos pavadinmą
\def \thesisTitleEN {Title of the doctoral dissertation}
%nosinės raidės \k{U}, \K{A}. Ė = \.{E}, jeigu prireiktų, bet nebūtina taip  naudoti
\def \thesisTitleLT {Daktaro disertacijos pavadinimas} 

%Turinio lentelių pervadinimas / Renaming of the thesis content tables
\def \tableOfContentsName {Table of Contents / Turinys}
\def \listOfTableName {List of Tables / Lentelių sąrašas}
\def \listOfFiguresName {List of Figures / Paveikslų sąrašas}

\def \thesisLanguage {EN} %Various settings
% [file: layout.tex, started: 18-March-2008]
%
% NOTES
%   This macro file contains page layout info for Ph.D. thesis
%
% CHANGES
%   2008.03.18	*	Started.

% ----- layout parameters ------------------------------------------------
\ifpdf
	\usepackage[dvips=false, pdftex=true, vtex=false]{geometry}
\else
	\usepackage[dvips=true, pdftex=false, vtex=false]{geometry}
\fi

% VU requirements 2021 m.
%puslapio dydis – B5 (17x24 cm),
%apatinė paraštė 2 cm, viršutinė paraštė 1,5 cm, kairė ir dešinė paraštės turi būti 2,5 cm,
%tarpai tarp eilučių – 1,15 intervalo,  (Rokas.: siuo metu paliktas defaultinis)
%pirmojo lygmens antraštė – 12 pt. Times New Roman; likusios paantraštės - 11 pt. Times New Roman,
%teksto šriftas – 11 pt. Times New Roman,
\geometry{%
 	b5paper,
    paperwidth=170mm, 
    paperheight=240mm,
 	bindingoffset=0pt,
 	centering,
 	hmargin=25mm,
    bottom=20mm,
    top=15mm%,
 	%includehead,
 	%includefoot
}  
  
\raggedbottom                           % height of text may vary per page

%\iffalse
\newlength{\oldparindent}
\newlength{\oldparskip}

\newcommand{\IZParagraph}{%
    \setlength{\oldparindent}{\parindent}
    \setlength{\oldparskip}{\parskip}
    \setlength{\parindent}{5mm}            % no indent
    \setlength{\parskip}{2ex plus 0.5ex minus 0.2ex}    % space between paragraphs
}

\newcommand{\restoreParagraph}{%
    \setlength{\parindent}{\oldparindent}
    \setlength{\parskip}{\oldparskip}
}
%\fi

%% VU requirements
%% 1.15 linespacing, bet paliktas defaultinis
%\linespread{1.15}
\usepackage{setspace}
%\onehalfspacing


\ifpdf
    \usepackage[pdftex]{hyperref,graphicx} % pdf references support

    \hypersetup{%
    	pdfauthor   = {\thesisAuthorName \ \thesisAuthorSurname},
    	pdftitle    = {\thesisTitleEN},
    	pdfsubject  = {PhD Thesis},
    	pdfkeywords = {\thesisTitleLT},
    	pdfcreator  = {LaTeX with hyperref package},
    	pdfproducer = {pdflatex},
    	breaklinks  = {true},
    	pdfstartview = {FitH},
    	bookmarksnumbered = {true},
    	bookmarksopen = {true},	
    	bookmarksopenlevel = 2,
    	plainpages={false},
    	pdfpagetransition={Dissolve},	
    	colorlinks = {true}, %defaultinis, padaro spalvotus stačiakampius ant visų linkų. Uždėjus true, bus tik spalvos ant skaicių. Turbut gražiau atrodo tik spalvos.
    	% Norint nespalvotu nuorodu į formules, literatura, etc, nustatome šias spalvas
    	linkcolor = {black}, %Vidinių nuorodų spalava
    	citecolor = {black}, %Citavimų spalva
    	filecolor = {black}, 
        urlcolor  = {black}, % Spalva išorinių nuorodų
    	% pilkos atspalviai:
    	% linkcolor = [rgb]{0.3, 0.3, 0.3},
    	%citecolor = [rgb]{0.6, 0.6, 0.6},
    	%filecolor = [rgb]{0.3, 0.3, 0.3},
    	%runcolor = [rgb]{0.3, 0.3, 0.3},
    }
\else
    \usepackage[dvips,ps2pdf]{hyperref,graphicx} % pdf references support
\fi

%%%%%%%%%%%%%%%%%%%%%%%%%%%%%%%%%%%%%%%%%%%%%%%%%%%%%%%%%%%%%%%%%%%%%%%%%%%%%%%%%%%%%%%%
%% Packages that are used in the document

\usepackage[immediate]{silence} % Package to silence some warnings
\WarningFilter[temp]{latex}{Command \underbar has changed.} % Silence sectsty package the warning
\WarningFilter[temp]{latex}{Command \underline has changed.} % Silence sectsty package the warning

%------ įkėliau iš preambule į santrauka
%\usepackage[L7x]{fontenc}
%\usepackage[utf8]{inputenc} % Accept different input encodings
\usepackage[T1]{fontenc} %Font encoding. The T1 font encoding is an 8-bit encoding and uses fonts that have 256 glyphs. 

\usepackage{times} %It does set \rmdefault to Times, \sfdefault to Helvetica, and \ttdefault to Courier

\usepackage{aeguill} %The package enables the user to add guillemets from several source (Polish cmr, Cyrillic cmr, lasy and ec) to the ae fonts.

\usepackage[lithuanian, english]{babel} %This package manages culturally-determined typographical (and other) rules for a wide range of languages.
\newcommand{\english}{\selectlanguage{english}}
\newcommand{\lithuanian}{\selectlanguage{lithuanian}}

\usepackage{amssymb}					% Extra AMS math symbols
\usepackage{amsfonts}					% Extra AMS math symbols
\usepackage{amsmath}					% AMS 
\usepackage[mathscr]{euscript}          % This file sets up some font shape definitions to use the Euler script symbols in math mode. 

\usepackage[table,dvipsnames]{xcolor}	% Colors support
\renewcommand{\topfraction}{0.85}	    %% 85% of page can contain figure
\renewcommand{\textfraction}{0.1}	    %% 10% of page can contain text
\renewcommand{\floatpagefraction}{0.75}	%% 75% of page should be figure to be only float page

\usepackage[square,comma,numbers,sort&compress]{natbib}	% Citations define numberic type citation, e.g, [1]
\setlength{\bibsep}{0ex}

\usepackage{fancyhdr}		    % Fancy headings and footers
\usepackage{sectsty}			% Change sections fonts, raises two warnings			

\setcounter{secnumdepth}{3}	    % Depth of enumerated sections

\ifpdf
	\usepackage{microtype}		% Experimental https://texdoc.org/serve/microtype/0
\else
\fi

\usepackage{makeidx}			% index package
\makeindex
% \usepackage{fix2col}			% fix two-column marks since 2015 not update use multicolumn instead. TODO delete
\makeatletter
\renewenvironment{theindex}
	{%
	\if@twocolumn
  	\@restonecolfalse
  \else
  	\@restonecoltrue
	\fi
	%
	\columnseprule \z@
	\columnsep 25\p@
	\twocolumn[\@makeschapterhead{\indexname}]%
	\@mkboth{\MakeUppercase\indexname}%
					{\MakeUppercase\indexname}%
	\thispagestyle{plain}\parindent\z@
	\parskip\z@ \@plus .3\p@\relax			
	\let\item\@idxitem	
	}
	{\if@restonecol\onecolumn\else\clearpage\fi}
%%

\renewcommand\@idxitem{\par\hangindent 30\p@}
\renewcommand\subitem{\@idxitem \hspace*{10\p@}}
\renewcommand\subsubitem{\@idxitem \hspace*{20\p@}} 

\renewcommand{\see}[2]{\emph{\seename}~{#1}}
\newcommand{\justsee}[2]{{#1}}


\makeatother	    % macros for dictionary like index			

\usepackage{palatino}			% IZ selection, font package

\usepackage{verbatim}           %IZ: multiline komentams 

\usepackage{paralist}           % Provides enumerate and itemize environments that can be used within paragraphs. https://ctan.org/pkg/paralist
\usepackage{tabularx}           % More advanced tables
\usepackage{longtable}          % For long tables used in acronyms, notations, vocabulary
\usepackage{booktabs}           % The package enhances the quality of tables. https://ctan.org/pkg/booktabs
\usepackage{multirow}           % Create tabular cells spanning multiple rows. https://ctan.org/pkg/multirow
\setlength{\belowcaptionskip}{5pt}

\usepackage{rotating}	        % Rotation tools, including rotated full-page floats https://ctan.org/pkg/rotating

\usepackage{epigraph}	        % A package for typesetting epigraphs. Epigraphs are the pithy quotations often found at the start (or end) of a chapter. https://ctan.org/pkg/epigraph

\usepackage{hyphenat}           %Disable/enable hypenation (žodžių perkėlimui įnaują       eilutę). By default is enabled in babel package. Used in title
 
\usepackage{pdfpages}           % Prikabina išorinius pdf failus prie latex pdf failo

\usepackage{array}              % Extending the array and tabular environments. https://ctan.org/pkg/array

\usepackage{float}              % Improved interface for floating objects. https://ctan.org/pkg/float

\usepackage{siunitx}   %paketas skaiciams ir SI vienetams vaizduoti. Galima ir nenaudoti, bet aukstesnis lygis, naudojamo pavyzdys \qty{67890}{\degree}
%\usepackage{textcomp}  %senas paketas, bet reikalingas kad nemestu kai kuriu siunitx klaidu
%\sisetup{load-configurations = abbreviations}

\setlength{\headheight}{19pt}  % Kažkokia problema su headheight

% Reikia sablonui. Gal galima ištrinti?
\usepackage[version=4]{mhchem} % Typeset chemical formulae/equations https://ctan.org/pkg/mhchem
\DeclareSIUnit \uM{\micro M}
\DeclareSIUnit \mM{\milli M}

\usepackage{tocloft}           % Control table of contents, figures, etc https://www.ctan.org/pkg/tocloft

\usepackage{titlesec}          % Select alternative section titles https://ctan.org/pkg/titlesec. Used for correct spacing and alignment of various titles

\usepackage{hologo}            % A collection of logos with bookmark support. https://ctan.org/pkg/hologo
\usepackage{cleveref}          % Intelligent cross-referencing. https://ctan.org/pkg/cleveref
\usepackage{tikz}              % Used to create figures

\usepackage{caption}           % Caption and subcaption packages allow to create subfigures and subtables
\usepackage{subcaption}        % see https://www.overleaf.com/learn/latex/How_to_Write_a_Thesis_in_LaTeX_(Part_3)%3A_Figures%2C_Subfigures_and_Tables#Subfigures

%%%%%%%%%%%%%%%%%%%%%%%%%%%%%%%%%%%%%%%%%%%%%%%%%%%%%%%%%%%%%%%%%%%%%%%%%%%%%%%%%%%%%%%%
%% Various document formatting redefinitions

% Set fonts according to the style / Nustatomi fontai pagal skyrių reikalavimus
\chapterfont{\fontsize{12}{13.5}\selectfont\normalfont\centering\MakeUppercase}
\sectionfont{\fontsize{12}{13.5}\selectfont\normalfont\centering}
\subsectionfont{\fontsize{11}{12}\selectfont\normalfont\centering}
\subsubsectionfont{\fontsize{11}{12}\selectfont\normalfont\centering}
\titlespacing*{\chapter} {0pt}{0pt}{13.5pt} %commaand from the titlesec package
\titlespacing*{\section} {0pt}{13.5pt}{13.5pt}
\titlespacing*{\subsection} {0pt}{13.5pt}{13.5pt}
\titlespacing*{\subsubsection} {0pt}{13.5pt}{13.5pt}

\titlelabel{\thetitle.\quad} %Add dot after chapter, section numbers
% The unnumbered definition follows
\titleformat{name=\chapter}[display]
    {\normalfont}{}{-14mm}{\thechapter. \quad\centering\normalfont\MakeUppercase}
\titleformat{name=\chapter,numberless}[display]
    {\normalfont}{}{-14mm}{\centering\normalfont\MakeUppercase}

%% TOC (Table of Contents) formatting
\tocloftpagestyle{plain}
\renewcommand{\cftchapleader}{\cftdotfill{\cftdotsep}} % for chapters adds dots in table of contents
\renewcommand{\cftchapfont}{\normalfont} %Change chapter font to normal in table of contents
\renewcommand{\cftchappagefont}{\normalfont} %Change chapter page number font to normal in table of contents

%Rename default Contents table name and change font size
\renewcommand{\cfttoctitlefont}{\fontsize{12}{13.5}\selectfont\normalfont\hfill}
\renewcommand{\cftlottitlefont}{\fontsize{12}{13.5}\selectfont\normalfont\hfill} %List of Tables
\renewcommand{\cftloftitlefont}{\fontsize{12}{13.5}\selectfont\normalfont\hfill} %List of Figures
%For centering title
\renewcommand{\cftaftertoctitle}{\hfill\ } 
\renewcommand{\cftafterlottitle}{\hfill\ } 
\renewcommand{\cftafterloftitle}{\hfill\ } 

%Formating TOC lists
\addto\captionsbritish{
    \renewcommand{\contentsname}{\MakeUppercase{\tableOfContentsName}}
    \renewcommand{\listtablename}{\MakeUppercase{\listOfTableName}}
    \renewcommand{\listfigurename}{\MakeUppercase{\listOfFiguresName}}
}
\addto\captionslithuanian{
    \renewcommand{\contentsname}{\MakeUppercase{\tableOfContentsName}}
    \renewcommand{\listtablename}{\MakeUppercase{\listOfTableName}}
    \renewcommand{\listfigurename}{\MakeUppercase{\listOfFiguresName}}
}

% Spacing after TOC titles
\setlength{\cftaftertoctitleskip}{13.5pt}
\setlength{\cftafterlottitleskip}{13.5pt}
\setlength{\cftafterloftitleskip}{13.5pt}

% Make correct spacing for all titles in TOC
\setlength{\cftbeforechapskip}{0pt}
\setlength{\cftbeforesecskip}{0pt}
\setlength{\cftbeforesubsecskip}{0pt}
\setlength{\cftbeforesubsubsecskip}{0pt}
\setlength{\cftbeforeparaskip}{0pt}
\setlength{\cftparskip}{5pt}

%Add dot after chapter, section, subsection number to fit dissetation format
\renewcommand{\cftchapaftersnum}{.}
\renewcommand{\cftsecaftersnum}{.}
\renewcommand{\cftsubsecaftersnum}{.}

%%%%%%%%%%%%%%%%%%%%%%%%%%%%%%%%%%%%%%%%%%%%%%%%%%%%%%%%%%%%%%%%%%%%%%%%%%%%%%%%%%%%%%%%
%% Additional macros

\def\interpti{\relax}%
\def\invardinti#1{%
     \trivlist\item[\hskip\labelsep{\bfseries#1}}%]
\def\atstumti{\endtrivlist}%\interpti}
\newif\ifRoman
\def\Fshape{\ifRoman\rmfamily\else\itshape\fi}
\newcommand{\trivardis}[3]{\invardinti{#1\ #2}\ (#3).]\Fshape}
\newcommand{\dvivardis}[2]{\invardinti{#1\ #2.}]\Fshape}
\makeatletter
\let\@opargbegintheorem\trivardis
\let\@begintheorem\dvivardis
\let\@endtheorem\atstumti
\newcommand{\newproclaim}[3]{\newenvironment{#1}{\global\finishedfalse
   \Romantrue\@thm{#2}{#3}}{\finish\atstumti}}
\makeatother
\newif\iffinished\finishedtrue
\newcommand{\finish}{\iffinished\else\ifhmode\nolinebreak\fi\nopagebreak%
   \qquad\ifhmode\nolinebreak\fi\nopagebreak%
   \ensuremath{\square}\global\finishedtrue\fi}

\newtheorem{claim}{Proposition}[section]
\newtheorem{corollary}[claim]{Corollary}
\newtheorem{theorem}[claim]{Theorem}
\newtheorem{lemma}[claim]{Lemma}
\newproclaim{definition}{claim}{Definition}
\newproclaim{example}{claim}{Example}
\newproclaim{remark}{claim}{Remark}
\newproclaim{notation}{claim}{Notation}
\newenvironment{proof}{\invardinti{Proof.}\global\finishedfalse]}{\finish\atstumti}
\newenvironment{princ}[1]{\invardinti{#1}]\itshape}{\atstumti}
\newenvironment{prule}[1]{\paragraph{#1}\equation}
                         {\endequation\addvspace{1ex plus.2ex minus .2ex}}


\newcommand{\allmoodsA}{\ensuremath{:\mkern-4mu-\allmoodslipsA}}
\newcommand{\allmoodsB}{\ensuremath{:\mkern-4mu-\mkern-3mu\allmoodslipsB}}
\newcommand{\allmoodslipsA}{\ensuremath{)\mkern-7.3mu|\mkern-7mu(}}
\newcommand{\allmoodslipsB}{\ensuremath{)\mkern-4.8mu|\mkern-4.4mu(}}

\newenvironment{Abstract}{%
	\begin{center}
		\textbf{Abstract}%
 	\end{center}
 	\small \it \begin{quote}
}
{\end{quote}}
% \makeatletter
\renewenvironment{theindex}
	{%
	\if@twocolumn
  	\@restonecolfalse
  \else
  	\@restonecoltrue
	\fi
	%
	\columnseprule \z@
	\columnsep 25\p@
	\twocolumn[\@makeschapterhead{\indexname}]%
	\@mkboth{\MakeUppercase\indexname}%
					{\MakeUppercase\indexname}%
	\thispagestyle{plain}\parindent\z@
	\parskip\z@ \@plus .3\p@\relax			
	\let\item\@idxitem	
	}
	{\if@restonecol\onecolumn\else\clearpage\fi}
%%

\renewcommand\@idxitem{\par\hangindent 30\p@}
\renewcommand\subitem{\@idxitem \hspace*{10\p@}}
\renewcommand\subsubitem{\@idxitem \hspace*{20\p@}} 

\renewcommand{\see}[2]{\emph{\seename}~{#1}}
\newcommand{\justsee}[2]{{#1}}


\makeatother %yra viršuje

%%%%%%%%%%%%%%%%%%%%%%%%%%%%%%%%%%%%%%%%%%%%%%%%%%%%%%%%%%%%%%%%%%%%%%%%%%%%%%%%%%%%%%%%
%% Useful commands

% \k{I} - uždeda nosines raides
% Unit conversion - https://tex.stackexchange.com/questions/8260/what-are-the-various-units-ex-em-in-pt-bp-dd-pc-expressed-in-mm