% [file: layout.tex, started: 18-March-2008]
%
% NOTES
%   This macro file contains page layout info for Ph.D. thesis
%
% CHANGES
%   2008.03.18	*	Started.
%   2024.09.02  *   Updated to fit Ph.D requirements

% ----- layout parameters ------------------------------------------------
\ifpdf
	\usepackage[dvips=false, pdftex=true, vtex=false]{geometry}
\else
	\usepackage[dvips=true, pdftex=false, vtex=false]{geometry}
\fi

% VU requirements from 2021.
%puslapio dydis – B5 (17x24 cm),
%apatinė paraštė 2 cm, viršutinė paraštė 1,5 cm, kairė ir dešinė paraštės turi būti 2,5 cm,
%tarpai tarp eilučių – 1,15 intervalo, 
%pirmojo lygmens antraštė – 12 pt. Times New Roman; likusios paantraštės - 11 pt. Times New Roman,
%teksto šriftas – 11 pt. Times New Roman,
\geometry{%
 	b5paper,
    paperwidth=170mm, 
    paperheight=240mm,
 	bindingoffset=0pt,
 	centering,
 	hmargin=25mm,
    bottom=20mm,
    top=15mm%,
 	%includehead,
 	%includefoot
}  
  
\raggedbottom                           % height of text may vary per page

%\iffalse
\newlength{\oldparindent}
\newlength{\oldparskip}

% Naudojama antram paragrafui atitraukti / It does intend of second paragraph
\newcommand{\IZParagraph}{%
    \setlength{\oldparindent}{\parindent}
    \setlength{\oldparskip}{\parskip}
    \setlength{\parindent}{7.5mm}            % no indent
    %\setlength{\parskip}{2ex plus 0.5ex minus 0.2ex}    % space between paragraphs
    \setlength{\parskip}{0ex plus 0.5ex minus 0.2ex}    % space between paragraphs
}

\newcommand{\restoreParagraph}{%
    \setlength{\parindent}{\oldparindent}
    \setlength{\parskip}{\oldparskip}
}
%\fi

%% VU requirements
%% 1.15 linespacing, bet paliktas defaultinis
\linespread{1.15}

% \usepackage{setspace}
% \setlength\parindent{0.75cm} %galima nusatyti visiems, bet naudojamas IZParagraph stilius
%\onehalfspacing
