\usepackage{ifpdf}

%%%%%%%%%%%%%%%%%%%%%%%%%%%%%%%%%%%%%%%%%%%%%%%%%%%%%%%%%%%%%%%%%%%%%%%%%%%%%%%%%%%%%%%%
%% Used to define various parameters used globally in the dissertation

%% The author name and surname
\def \thesisAuthorName {Vardas}
\def \thesisAuthorSurname {Pavardė}

%% DOI number of the dissertation / DOI numeris (suteikiamas atsiuntus disertaciją spausdinti)
\def \thesisDOI {https://doi.org/}

%% If you do not have ORCID please create it using https://orcid.org / Jeigu ORCID dar neturite, nemokamai galite susikurti https://orcid.org/)
\def \thesisORCID {https://orcid.org/0000-0001-2345-6789} 

%% Year of start Ph.D and  year of writing this dissertation
\def \thesisPreparationStartYear {2020}
\def \thesisYear {2024}

%% The title of dissetaion in English (thesisTitleEN) and in Lithuanian (thesisTitleLT)
%% Įveskite daktaro disertacijos pavadinmą angliškai (thesisTitleEN) ir lietuviškai (thesisTitleLT)
\def \thesisTitleEN {Title of the doctoral dissertation}
\def \thesisTitleLT {Daktaro disertacijos pavadinimas} 

%% Renaming of Table of Contents, List of Tables, and List of Figures
%% Turinio lentelių pervadinimas
\def \tableOfContentsName {Table of Contents / Turinys}
\def \listOfTableName {List of Tables / Lentelių sąrašas}
\def \listOfFiguresName {List of Figures / Paveikslų sąrašas}

%% Define editors for Enlish version and Lithuanian version
%% Disertacijos ir santraukos redaktoriai
\def \thesisEditorEN {Name Surname}
\def \thesisEditorLT {Name Surname}
 %Various settings
% [file: layout.tex, started: 18-March-2008]
%
% NOTES
%   This macro file contains page layout info for Ph.D. thesis
%
% CHANGES
%   2008.03.18	*	Started.

% ----- layout parameters ------------------------------------------------
\ifpdf
	\usepackage[dvips=false, pdftex=true, vtex=false]{geometry}
\else
	\usepackage[dvips=true, pdftex=false, vtex=false]{geometry}
\fi

% VU requirements 2021 m.
%puslapio dydis – B5 (17x24 cm),
%apatinė paraštė 2 cm, viršutinė paraštė 1,5 cm, kairė ir dešinė paraštės turi būti 2,5 cm,
%tarpai tarp eilučių – 1,15 intervalo,  (Rokas.: siuo metu paliktas defaultinis)
%pirmojo lygmens antraštė – 12 pt. Times New Roman; likusios paantraštės - 11 pt. Times New Roman,
%teksto šriftas – 11 pt. Times New Roman,
\geometry{%
 	b5paper,
    paperwidth=170mm, 
    paperheight=240mm,
 	bindingoffset=0pt,
 	centering,
 	hmargin=25mm,
    bottom=20mm,
    top=15mm%,
 	%includehead,
 	%includefoot
}  
  
\raggedbottom                           % height of text may vary per page

%\iffalse
\newlength{\oldparindent}
\newlength{\oldparskip}

\newcommand{\IZParagraph}{%
    \setlength{\oldparindent}{\parindent}
    \setlength{\oldparskip}{\parskip}
    \setlength{\parindent}{5mm}            % no indent
    \setlength{\parskip}{2ex plus 0.5ex minus 0.2ex}    % space between paragraphs
}

\newcommand{\restoreParagraph}{%
    \setlength{\parindent}{\oldparindent}
    \setlength{\parskip}{\oldparskip}
}
%\fi

%% VU requirements
%% 1.15 linespacing, bet paliktas defaultinis
%\linespread{1.15}
\usepackage{setspace}
%\onehalfspacing



\ifpdf
\usepackage[pdftex]{hyperref,graphicx} % pdf references support

\hypersetup{%
	pdfauthor   = {\thesisAuthorName \ \thesisAuthorSurname},
	pdftitle    = {},
	pdfsubject  = {PhD Thesis},
	pdfkeywords = {\thesisTitleEN},
	pdfcreator  = {LaTeX with hyperref package},
	pdfproducer = {pdflatex},
	breaklinks  = {true},
	pdfstartview = {FitH},
	bookmarksnumbered = {true},
	bookmarksopen = {true},	
	bookmarksopenlevel = 2,
	plainpages={false},
	pdfpagetransition={Dissolve},	
	colorlinks = {true}, %defaultinis, padaro spalvotus staciakampius ant visu linku. Uzdejus true, bus tik spalvos ant skaiciu. Turbut graziau atrodo tik spalvos.
	%Norint nespalvotu nuorodu i formules, literatura, etc:
	%linkcolor = {black},
	%citecolor = {black},
	%filecolor = {black},
	%pilkos atspalviai:
	%linkcolor = [rgb]{0.3, 0.3, 0.3},
	%citecolor = [rgb]{0.6, 0.6, 0.6},
	%filecolor = [rgb]{0.3, 0.3, 0.3},
	%runcolor = [rgb]{0.3, 0.3, 0.3},
}
\else
\usepackage[dvips,ps2pdf]{hyperref,graphicx} % pdf references support
\fi

% ----- packages ---------------------------------------------------------
\renewcommand{\contentsname}{Table of Contents}
%\usepackage[british]{babel}		% british-english style
%------ ikeliau is preambule-santrauka
%\usepackage[L7x]{fontenc}
%\usepackage[utf8]{inputenc}
\usepackage[T1]{fontenc}
\usepackage{aeguill}
\usepackage[lithuanian, british]{babel}
\usepackage{ogonek}  %komandai \k kuri uzdeda kableli ant raidziu
\newcommand{\british}{\selectlanguage{british}}
\newcommand{\lithuanian}{\selectlanguage{lithuanian}}
%-----
%Sprendimas į raidei
\usepackage{newunicodechar}
%\newunicodechar{Į}{\k{I}}
%\newunicodechar{į}{\k{i}}

\usepackage{amssymb}					% extra AMS math symbols
\usepackage{amsfonts}					% extra AMS math symbols
\usepackage{amsmath}					% AMS 
\usepackage[mathscr]{euscript}

\usepackage{xcolor}				% colors support
\renewcommand{\topfraction}{0.85}	%% 85% of page can contain figure
\renewcommand{\textfraction}{0.1}	%% 10% of page can contain text
\renewcommand{\floatpagefraction}{0.75}	%% 75% of page should be figure to be only float page

\usepackage[square,comma,numbers,sort&compress]{natbib}	% citations
\setlength{\bibsep}{0ex}

\usepackage{fancyhdr}		% fancy headings and footers

\usepackage{sectsty}			% change sections fonts									
\setcounter{secnumdepth}{3}	% depth of enumerated sections

\ifpdf
	\usepackage{microtype}		% experimental
\else
\fi

\usepackage{makeidx}			% index package
\makeindex								% generate index entries
%\usepackage{fix2col}			% fix two-column marks
%%%%%%%%%%%%%%%%%%%%%%%%%%%%%%%%%%%%%%%%%%%%%%%%%%%%%%%%%%%%%%%%%%%%%%%%%%%%%%%%%%%%%%%%%%%%%%
%% [file: index.tex, started: 25-Aug-2005]
%%
%% PhD Thesis - top level LaTeX source file.
%%
%% DESCRIPTION
%%   This file includes Index chapter of the PhD Thesis. 
%%   USe command \index{keyword} to add "keyword" to the index
%%
%% CHANGES
%%   2005.08.25  *  Started.
%%   2008.03.18  *  Adapted to IZ.
%%   2024.08.30  *  Fixed to fit VU dissertation template

\chapter{Index}
\label{chapter:Index}
% \addtocontents{toc}{\protect\enlargethispage{2\baselineskip}}
\phantomsection
% \addcontentsline{toc}{chapter}{\numberline{}Index} % \numberline{} atitraukia į šalį
\addcontentsline{toc}{chapter}{\MakeUppercase{Index}}

%% Show index items in two-column style sorted by name
%To add something to the index please use command \index{keyword}
{\raggedright
    \printindex
}
	% macros for dictionary like index			

\usepackage{palatino}				% IZ selection
\usepackage[T1]{fontenc}		% modern tex fonts

\usepackage{verbatim} %IZ: multiline komentams 

\usepackage{paralist}
\usepackage{tabularx}
\usepackage{booktabs}
\usepackage{multirow}
\setlength{\belowcaptionskip}{5pt}

\usepackage{rotating}	

\usepackage{epigraph}	

\usepackage{hyphenat}

\usepackage{pdfpages}   %Prikabina isorinius pdfus prie latex pdfo

\usepackage{array}
\usepackage{float}

\usepackage{siunitx}   %paketas skaiciams ir SI vienetams vaizduoti. Galima ir nenaudoti, bet aukstesnis lygis
\usepackage{textcomp}  %senas paketas, bet reikalingas kad nemestu kai kuriu siunitx klaidu
\sisetup{load-configurations = abbreviations}
\setlength{\headheight}{19pt}  %kazkokia problema su headheight

%Reikia sablonui. Galima istrinti
\usepackage[version=4]{mhchem}
\DeclareSIUnit \uM{\micro M}
\DeclareSIUnit \mM{\milli M}




% ----- macros ---------------------------------------------------------
\input{macros/iz_macros.tex}
\newcommand{\IZheading}[1]{\scriptsize{\textit{#1}}}

\pagestyle{fancy}
\renewcommand{\chaptermark}[1]{\markboth{\thechapter.\ #1}{}}
\renewcommand{\sectionmark}[1]{\markright{\thesection.\ #1}}
% \renewcommand\thechapter{\arabic{section}}
% \renewcommand\thesection{\arabic{section}}

% \def\thechapter{\arabic{chapter}}

% \newcommand{\IZThesisHeadings}{%
% 	\fancyhead{}\fancyhead[LE]{\IZheading{\IZheading{\leftmark}}}\fancyhead[RO]{\IZheading{\rightmark}}
% 	\fancyfoot{}
% 	\fancyfoot[c]{\thepage}
% 	\fancypagestyle{plain}{%
% 		\fancyhf{}
% 		\fancyfoot[c]{\thepage}
% 		\renewcommand{\headrulewidth}{0pt}
% 	}
% }

\makeatletter
\def\cleardoublepage{%
	\clearpage\if@twoside \ifodd\c@page\else%
	\hbox{}%
	\thispagestyle{empty}%
	\newpage%
	\if@twocolumn\hbox{}\newpage\fi\fi\fi
}

\makeatother
    
\makeatletter	
\newcommand{\itemhdr}[1]{\par\hangindent 30\p@\relax{#1}\markboth{#1}{#1}}
\makeatother

\newcommand{\indexmarks}{%
	\fancyhead[R]{\IZheading{\leftmark}}
	\fancyhead[L]{\IZheading{\rightmark}}
}

% documentation is available here https://www.overleaf.com/learn/latex/Headers_and_footers#Notes_on_the_commands_used
\newcommand{\parammarks}[1]{
	\fancyhead{}                      %clears the settings for the headers
	\fancyhead[RO,LE]{l}               %uses the header locations RO (RightOdd) and LE (LeftEven) to                                    place the content
	\fancyhead[RO,LE]{\IZheading{#1}} %uses the header locations RO (RightOdd) and LE (LeftEven) to                                    place the content \IZheading{#1}
	\fancyfoot{}                      %clears the settings for the footers
	\fancyfoot[c]{\thepage}           %uses the footer locations center (c) to place the content \                                        of \thepage
	\fancypagestyle{plain}{%  just add page numer at the center of page footer 
		\fancyhf{}	
		\fancyfoot[c]{\thepage}
	\renewcommand{\headrulewidth}{0pt}%
	}
}
