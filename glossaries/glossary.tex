%%%%%%%%%%%%%%%%%%%%%%%%%%%%%%%%%%%%%%%%%%%%%%%%%%%%%%%%%%%%%%%%%%%%%%%%%%%%%%%%%%%%%%%%%%%%%%
%% A glossary is a specialized list of terms and definitions often found at the end of a book, article, 
%% or in a field of study. It %% is an alphabetical list of words pertaining to a specific subject, with explanations; 
%% a reference tool to aid understanding of specialized language. 

%%%%%%%%%%%%%%%%%%%%%%%%%%%%%%%%%%%%%%%%%%%%%%%%%%%%%%%%%%%%%%%%%%%%%%%%%%%%%%%%
% Glossary items
%%%%%%%%%%%%%%%%%%%%%%%%%%%%%%%%%%%%%%%%%%%%%%%%%%%%%%%%%%%%%%%%%%%%%%%%%%%%%%%%

\newglossaryentry{base learner}{
  name={base learner},
  description={bazinis klasifikatorius.}
}

\newglossaryentry{latex}{
    name=latex,
    plural=latexes,   % Not obligatory, but here you can define plural form of name text.
    description={\hologo{LaTeX} (pronounced “LAY-tek” or “LAH-tek”) is a tool for typesetting professional-looking documents. However, LaTeX’s mode of operation is quite different to many other document-production applications you may have used, such as Microsoft Word or LibreOffice Writer: those “WYSIWYG” tools provide users with an interactive page into which they type and edit their text and apply various forms of styling.}
}

%%%%%%%%%%%%%%%%%%%%%%%%%%%%%%%%%%%%%%%%%%%%%%%%%%%%%%%%%%%%%%%%%%%%%%%%%%%%%%%%%%%%%%%%%%%%%%
%% Glossary printing
%% Do not alter the content below

\setglossarysection{chapter}
\setglossarystyle{gls-style}
\printglossary[type=main,title={\MakeUppercase{Glossary - Žodynėlis}}]
% \paragraph{Note:} Only used in the thesis text glossary terms will appear in the pdf document of the thesis



%%%%%%%%%%%%%%%%%%%%%%%%%%%%%%%%%%%%%%%%%%%%%%%%%%%%%%%%%%%%%%%%%%%%%%%%%%%%%%%%%%%%%%%%%%%%%%
%% For manual entering, please use this code

% \chapter*{Glossary - \v{Z}odyn\.elis}
% \label{cha:glossary}
% \addcontentsline{toc}{chapter}{\MakeUppercase{Glossary - \v{Z}odyn\.elis}}

% \noindent
%  \begin{longtable}[l]{ p{4cm} p{7cm} } %Todal available width is 11cm
%     base learner   &  bazinis klasifikatorius \\
%     baseline       &  bazinis metodas \\
%     change point   &  poky\v{c}io ta\v{s}kas  \\
%     concept drift  &  koncepcijos pokytis \\
%     context aware  &  kontekstinis  \\
%     data mining    &  duomen\k{u} gavyba \\ 
%     data source    &  duomen\k{u} \v{s}altinis \\
%     gradual drift  &  palaipsnis pokytis  \\
%     instance       &  vektorius \\
%     instance based learning &  mokymas pagal vektorius \\ 
%     label          &  klas\.e \\
%     moving average &  slenkantis vidurkis \\ 
%     peer methods   &  lyginamieji metodai  \\
%     recurring concepts  &  pasikartojantis pokytis (pasikartojan\v{c}ios koncepcijos)  \\
%     sequential learning &  mokymas paeiliui  \\
%     source         &  \v{s}altinis  \\
%     sudden drift   &  staigus pokytis  \\
%     supervised learning   &  mokymas su mokytoju  \\
%     training window       &  mokymo langas  \\
%     unsupervised learning &  mokymasis  \\
%     \label{tab:vocabulary}
% \end{longtable}